\providecommand{\thebibpath}{..}
\makeatletter\def\input@path{{\thebibpath/}{.}}\makeatother
\documentclass[main.tex]{subfiles}
\begin{document}

The hazard assessment form completed in 2016 was on the whole, accurate for the project.
There were two cases of unforseen hazards.

The first was the need to operate a drill press to make modifications to the robot, which was a task not predicted in the initial assessment.
Safely glasses were warn, and materials were clamped as well as possible, to minimize risk.
No safety issues arose from this work.

The second unforseen hazard was a thermal issue originating from an electrical one.
An electrical short with no protective fuse occurred on the hardware, due to poor design from work prior to the beginning of my project.
This was not possible to predict at the hand-in for the hazard assessment, as there had not been time to review the electrical design.

This incident posed an immediate safety concern -- shorted LiPo batteries can explode in nasty ways, which is why when changing, the battery is kept in a firesafe bag.
Using this bag is not viable during robot operation (nor is this a commmon approach when using these batteries).
In order to avoid such an explosive scenario, the power cables were quickly unplugged, but not before they rapidly melted (both insulator and conductor).
The circuit was successfully broken, but at the cost of a burn on a fingertip (which has since healed).

To avoid this incident repeating itself, a fuse was added to the system, and a set of wire cutters kept at hand to allow for a quick approach to cutting the power supply.

\end{document}