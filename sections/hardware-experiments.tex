\providecommand{\thebibpath}{..}
\makeatletter\def\input@path{{\thebibpath/}{.}}\makeatother
\documentclass[main.tex]{subfiles}
\begin{document}

	\begin{figure}[b!]
		\centering
		\begin{minipage}{0.5\linewidth - 2em}
			% This file was created by matlab2tikz.
%
%The latest updates can be retrieved from
%  http://www.mathworks.com/matlabcentral/fileexchange/22022-matlab2tikz-matlab2tikz
%where you can also make suggestions and rate matlab2tikz.
%
\definecolor{mycolor1}{rgb}{0.00000,0.44700,0.74100}%
\definecolor{mycolor2}{rgb}{0.85000,0.32500,0.09800}%
\definecolor{mycolor3}{rgb}{0.92900,0.69400,0.12500}%
%
\begin{tikzpicture}[%
trim axis left,
trim axis right
]

\begin{axis}[%
width=\linewidth,
height=5cm,
at={(0\linewidth,0cm)},
scale only axis,
xmin=0.5,
xmax=50.5,
xlabel style={font=\color{white!15!black}},
xlabel={\strut iteration number},
ymin=-0.1,
ymax=2.6,
ylabel style={font=\color{white!15!black}},
ylabel={time upright / $\si{\second}$},
axis background/.style={fill=white},
axis x line*=bottom,
axis y line*=left,
xmajorgrids,
ymajorgrids,
legend style={at={(0.03,0.97)}, anchor=north west, legend cell align=left, align=left, draw=white!15!black}
]
\addplot [color=mycolor1]
  table[row sep=crcr]{%
1	0.4\\
2	0.6\\
3	0.6\\
4	0.5\\
5	0.35\\
6	0.55\\
7	0.8\\
8	0.5\\
9	0.4\\
10	0.7\\
11	0.2\\
12	0.5\\
13	0.65\\
14	0.55\\
15	0.6\\
16	0.45\\
17	0.25\\
18	0.2\\
19	0.45\\
20	0.35\\
21	0.8\\
22	0.6\\
23	0.4\\
24	0.5\\
25	0.3\\
26	0.55\\
27	0.35\\
28	0.65\\
29	0.65\\
30	0.55\\
31	0.6\\
32	0.45\\
33	0.65\\
};
\addlegendentry{experiment 1}

\addplot [color=mycolor2]
  table[row sep=crcr]{%
1	0.4\\
2	0.6\\
3	0.6\\
4	0.5\\
5	0.35\\
6	0.55\\
7	0.8\\
8	0.5\\
9	0.4\\
10	0.7\\
11	0.45\\
12	0.5\\
13	0.35\\
14	0.45\\
15	0.25\\
16	0.4\\
17	0.4\\
18	0.4\\
19	0.5\\
20	0.4\\
21	0.4\\
22	0.35\\
23	0.7\\
24	0.65\\
25	0.6\\
26	0.45\\
27	0.55\\
28	0.6\\
29	0.6\\
30	0.45\\
31	0.6\\
32	0.5\\
33	0.75\\
34	0.45\\
35	0.45\\
36	0.45\\
37	0.45\\
38	0.5\\
39	0.6\\
40	0.3\\
41	0.45\\
42	0.55\\
43	0.6\\
44	0.5\\
45	0.45\\
46	0.7\\
47	0.4\\
48	0.4\\
};
\addlegendentry{experiment 2}

\addplot [color=mycolor3]
  table[row sep=crcr]{%
1	0.1\\
2	0.15\\
3	0.1\\
4	0.05\\
5	0.1\\
6	0.1\\
7	0.15\\
8	0.1\\
9	0.1\\
10	0.1\\
11	0.15\\
12	0.2\\
13	0.2\\
14	0.3\\
15	0.2\\
16	0.2\\
17	0.25\\
18	0.3\\
19	0.35\\
20	0.35\\
21	0.25\\
22	0.35\\
23	0.3\\
24	0.25\\
25	0.6\\
26	1\\
27	0.6\\
28	0.45\\
29	0.45\\
30	0.7\\
31	0.75\\
32	0.9\\
33	0.45\\
34	0.8\\
35	1.15\\
36	0.95\\
37	0.9\\
38	1.6\\
39	1.5\\
40	1.6\\
41	1.55\\
42	2.4\\
43	1.35\\
44	1.6\\
45	2.15\\
46	1.3\\
47	2.45\\
48	1.65\\
49	1.55\\
50	1.2\\
};
\addlegendentry{simulated}

\end{axis}
\end{tikzpicture}%
		\end{minipage}\hfill
		\begin{minipage}{0.5\linewidth - 2em}
			% This file was created by matlab2tikz.
%
%The latest updates can be retrieved from
%  http://www.mathworks.com/matlabcentral/fileexchange/22022-matlab2tikz-matlab2tikz
%where you can also make suggestions and rate matlab2tikz.
%
\definecolor{mycolor1}{rgb}{0.00000,0.44700,0.74100}%
\definecolor{mycolor2}{rgb}{0.85000,0.32500,0.09800}%
\definecolor{mycolor3}{rgb}{0.92900,0.69400,0.12500}%
%
\begin{tikzpicture}[%
trim axis left,
trim axis right
]

\begin{axis}[%
width=\linewidth,
height=5cm,
at={(0\linewidth,0cm)},
scale only axis,
xmin=0,
xmax=40,
xlabel style={font=\color{white!15!black}},
xlabel={total experience / $\si{\second}$},
ymin=-0.1,
ymax=2.6,
ylabel style={font=\color{white!15!black}},
ylabel={\strut time upright / $\si{\second}$},
axis background/.style={fill=white},
axis x line*=bottom,
axis y line*=left,
xmajorgrids,
ymajorgrids,
legend style={at={(0.03,0.97)}, anchor=north west, legend cell align=left, align=left, draw=white!15!black}
]
\addplot [color=mycolor1]
  table[row sep=crcr]{%
0.4	0.4\\
1	0.6\\
1.6	0.6\\
2.1	0.5\\
2.45	0.35\\
3	0.55\\
3.8	0.8\\
4.3	0.5\\
4.7	0.4\\
5.4	0.7\\
5.6	0.2\\
6.1	0.5\\
6.75	0.65\\
7.3	0.55\\
7.9	0.6\\
8.35	0.45\\
8.6	0.25\\
8.8	0.2\\
9.25	0.45\\
9.6	0.35\\
10.4	0.8\\
11	0.6\\
11.4	0.4\\
11.9	0.5\\
12.2	0.3\\
12.75	0.55\\
13.1	0.35\\
13.75	0.65\\
14.4	0.65\\
14.95	0.55\\
15.55	0.6\\
16	0.45\\
16.65	0.65\\
};
\addlegendentry{experiment 1}

\addplot [color=mycolor2]
  table[row sep=crcr]{%
0.4	0.4\\
1	0.6\\
1.6	0.6\\
2.1	0.5\\
2.45	0.35\\
3	0.55\\
3.8	0.8\\
4.3	0.5\\
4.7	0.4\\
5.4	0.7\\
5.85	0.45\\
6.35	0.5\\
6.7	0.35\\
7.15	0.45\\
7.4	0.25\\
7.8	0.4\\
8.2	0.4\\
8.6	0.4\\
9.1	0.5\\
9.5	0.4\\
9.9	0.4\\
10.25	0.35\\
10.95	0.7\\
11.6	0.65\\
12.2	0.6\\
12.65	0.45\\
13.2	0.55\\
13.8	0.6\\
14.4	0.6\\
14.85	0.45\\
15.45	0.6\\
15.95	0.5\\
16.7	0.75\\
17.15	0.45\\
17.6	0.45\\
18.05	0.45\\
18.5	0.45\\
19	0.5\\
19.6	0.6\\
19.9	0.3\\
20.35	0.45\\
20.9	0.55\\
21.5	0.6\\
22	0.5\\
22.45	0.45\\
23.15	0.7\\
23.55	0.4\\
23.95	0.4\\
24.65	0.7\\
25.15	0.5\\
};
\addlegendentry{experiment 2}

\addplot [color=mycolor3]
  table[row sep=crcr]{%
0.1	0.1\\
0.25	0.15\\
0.35	0.1\\
0.4	0.05\\
0.5	0.1\\
0.6	0.1\\
0.75	0.15\\
0.85	0.1\\
0.95	0.1\\
1.05	0.1\\
1.2	0.15\\
1.4	0.2\\
1.6	0.2\\
1.9	0.3\\
2.1	0.2\\
2.3	0.2\\
2.55	0.25\\
2.85	0.3\\
3.2	0.35\\
3.55	0.35\\
3.8	0.25\\
4.15	0.35\\
4.45	0.3\\
4.7	0.25\\
5.3	0.6\\
6.3	1\\
6.9	0.6\\
7.35	0.45\\
7.8	0.45\\
8.5	0.7\\
9.25	0.75\\
10.15	0.9\\
10.6	0.45\\
11.4	0.8\\
12.55	1.15\\
13.5	0.95\\
14.4	0.9\\
16	1.6\\
17.5	1.5\\
19.1	1.6\\
20.65	1.55\\
23.05	2.4\\
24.4	1.35\\
26	1.6\\
28.15	2.15\\
29.45	1.3\\
31.9	2.45\\
33.55	1.65\\
35.1	1.55\\
36.3	1.2\\
};
\addlegendentry{simulated}
\legend{};

\end{axis}
\end{tikzpicture}%
		\end{minipage}
		\caption{Lack of learning progress in reality}
		\label{fig:learning-compare}
		\medskip
		\small
		Note the the first 10 rollouts were reused between the two experiments.
	\end{figure}

	With all of the obvious problems corrected, it was finally time to apply the \textsc{Pilco} procedure.
	$10$ rollouts were performed under an affine controller with parameters
	\begin{align}
		\bm{b} &= \bm{0}, &
		W &=
			\frac{1}{\SI{15}{\degree}}
			\unitv{\psi}^T\unitv{\tau_w}
			+ \frac{1}{\SI{90}{\degree\per\second}}
			\unitv{\dot\phi}^T\unitv{\tau_t}\,,
	\end{align}
	and then remaining rollouts were performed with learnt controllers.
	Performance (measured by time upright) is compared with that of the simulated robot in \cref{fig:learning-compare}.
	Performance in reality\footnotemark is better than in simulation for the first 20 trials, due to perhaps:
	\begin{itemize}[noitemsep]
		\item A mismatch in the simulation model (\cref{table:mechanical}), specifically too small a frame moment of inertia -- these parameters are also are the least accurately estimated
		\item A better (non-random) initial controller, that does not actively throw the robot into the ground.
		\item A narrowed initial state distribution -- in simulation, the robot is started with initial velocities.
	\end{itemize}
	\footnotetext{Previous work~\cite{aleksi} concluded that this longer time upright was due to human reaction time in releasing the robot, but the work in \cref{sec:switch} makes this no longer relevant.}
	However, after this point, the simulation begins to learn a better controller, whereas the experimental run do not.
	More alarmingly, this trea

	\subsection{Correcting the initial state}
	It was realized that after the improvements made in \cref{sec:acc:orient}, the initial state of the robot would no longer always be a vector of zeros. The symptom of this problem is shown in \todo{}.
	As such, it became necessary to compute a distribution of starts states.
	For simplicity, a Gaussian with a sparse diagonal covariance matrix is fitted to the data -- sparse because fitting a dense matrix would need far too many data points to not be degenerate. Future work could apply a relatively simple Bayesian learning approach here, in order to start with a prior.

	Experiment 2 from \label{fig:learning-compare} was conducted after these changes were made.

	With this in place, it becomes easy to test our earlier hypothesis about the narrower initial state distribution, by comparing this fitted distribution against the one used in simulation -- indeed, it proves to be false, with the distribution in reality being about $5\times$ wider\footnotemark.

	\footnotetext{Except in the yaw axis, which is always zero on the hardware, and in the position axes, due to a small bug that was then fixed}

\bib

\end{document}