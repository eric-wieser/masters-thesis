\providecommand{\thebibpath}{..}
\makeatletter\def\input@path{{\thebibpath/}{.}}\makeatother
\documentclass[main.tex]{subfiles}
\begin{document}

\section{Automatic Differentiation}

	Another pattern that plagues not only the \textsc{Pilco} codebase, but machine learning software in general, is the manual computation of gradients.
	Manually deriving and implementing gradients is bad practice because, to quote an opinion piece \cite{ad-criminal}, \enquote{Important ideas get obscured by the details}, and worse, \enquote{[it] restricts the community to only using computational structures we are capable of manually deriving gradients for}.
	This goes on to recommend the use of \enquote{Automatic Differentiation} (AD), which computes the gradients in the same way that a manual implementation does.

	A brief investigation was conducted into applying AD to the Matlab \textsc{Pilco} code, and it was found to be a large enough task that it probably warrants its own project. Additionally, the only Matlab implementation of reverse-AD (ADiMat) is not well-maintained, and was found to crash on the author's machine.
	Therefore, adoption of AD would likely require a switch to a different language.
	The author believes that switching to Python would be the way to go\footnotemark, noting that the \texttt{autograd} \cite{autograd} package supports AD of the Cholesky factorization, a key operation within \textsc{Pilco}.

	\footnotetext{Although the author is biased here, as an active maintainer of \textit{the} numerical python package, \texttt{numpy}}

	\section{Quadratic control}

	It seems than in doing the work needed to switch to the small unicycle, and the pause put in the project during that transition, much of the work by \citeauthor{queiro} was forgotten.
	In particular, they remark that a linear controller is insufficient in the sense of reaching a target position in \cite[fig.~8]{queiro}, and thus used a quadratic controller.
	We can go further, and prove that even symmetric upright balancing is not possible with a linear controller.

	Consider a stationary unicycle, with all states at the origin apart from pitch ($\psi$) and roll ($\theta$), and the action needed to stabilize it.
	Clearly, pitch can be stabilized by driving forward when leaning forward, and backwards when learning backwards.
	The decisions on how to control the turntable are more complex, and shown pictorially in \cref{fig:balancing}.
	The wheel controller in (a)(i) can be described by $\tau_w = k\psi$, a linear controller.
	However, the turntable controller in (b)(i) requires a controller of the form $\tau_t = k\psi\theta$, a quadratic controller.
	While some success was shown using the linear controller in this and previous reports, this success can only ever hold when the robot stays within one half-plane of the pitch-roll state-space -- that is, always remains tilted forwards, or to one side.
	Indeed, (b)(ii) only matches (b)(i) in the half-plane where $\psi > 0$, suggesting the controller was learnt primarily from forward falls. (b)(ii) shows the opposite, that the harware learnt from backwards falls.

	\newenvironment{ctrlaxis}{%
		\begin{axis}[%
			width=\linewidth,
			height=0.75\linewidth,
			scale only axis,
			xmin=-pi/2,xmax=pi/2,
			ymin=-3*pi/8,ymax=3*pi/8,
			y dir=reverse,
			xtick={-0.7853981, 0, 0.7853981},
			ytick={-0.7853981, 0, 0.7853981},
			ylabel={$\theta$},
			xlabel={$\psi$},
			xticklabels={{$-\frac{\pi}{4}$},{},{$\frac{\pi}{4}$}},
			yticklabels={{$-\frac{\pi}{4}$},{},{$\frac{\pi}{4}$}},
			axis lines=middle,
			axis on top=true
		]%
	}{%
		\end{axis}%
	}
	\newcommand{\drawge}{-- (rel axis cs:1,0) -- (rel axis cs:1,1) -- (rel axis cs:0,1) \closedcycle}
	\newcommand{\drawle}{-- (rel axis cs:1,1) -- (rel axis cs:1,0) -- (rel axis cs:0,0) \closedcycle}
	\newcommand\mybox[2][]{\tikz[overlay]\node[inner sep=2pt, anchor=text, rectangle, rounded corners=1mm,#1] {#2};\phantom{#2}}


	\newlength{\ctlpadding}
	\setlength{\ctlpadding}{2em}

	\begin{figure}
		\singlespacing
		\centering
		\begin{subfigure}[t]{\linewidth}
      \makebox[0pt]{\rotatebox[origin=c]{90}{
        (a) wheel, $\tau_w$
      }\hspace*{2em}}%
			\begin{subfigure}{(\linewidth - 2\ctlpadding)/3}
				\centering
				\begin{tikzpicture}
					\begin{ctrlaxis}
						\fill[green!25]
							(current axis.above origin) rectangle (current axis.south east);
						\fill[red!25]
							(current axis.above origin) rectangle (current axis.south west);
					\end{ctrlaxis}
				\end{tikzpicture}
			\end{subfigure}\hfill
			\begin{subfigure}{(\linewidth - 2\ctlpadding)/3}
				\centering
				\begin{tikzpicture}
					\begin{ctrlaxis}
						% cw
						\addplot[domain=-pi:pi, samples=2, draw=none, fill=green!25] {x*-9.617634 + -0.187629} \drawge;
						\addplot[domain=-pi:pi, samples=2, draw=none, fill=red!25] {x*-9.617634 + -0.187629} \drawle;
						\addplot[domain=-pi:pi, samples=2] {x*-9.617634 + -0.187629};
					\end{ctrlaxis}
				\end{tikzpicture}
			\end{subfigure}\hfill
			\begin{subfigure}{(\linewidth - 2\ctlpadding)/3}
				\centering
				\begin{tikzpicture}
					\begin{ctrlaxis}
						% cw
						\addplot[domain=-pi:pi, samples=2, draw=none, fill=green!25] {x*1.267563 + -0.008781} \drawle;
						\addplot[domain=-pi:pi, samples=2, draw=none, fill=red!25] {x*1.267563 + -0.008781} \drawge;
						\addplot[domain=-pi:pi, samples=2] {x*1.267563 + -0.008781};
					\end{ctrlaxis}
				\end{tikzpicture}
			\end{subfigure}%
			\label{fig:balancing:wheel}%
		\end{subfigure}\\
		%
		\vskip-0.12em
		\vskip\ctlpadding
		%
		\begin{subfigure}[t]{\linewidth}
      \makebox[0pt]{\rotatebox[origin=c]{90}{
        (b) turntable, $\tau_t$
      }\hspace*{2em}}%
			\centering
			\begin{subfigure}{(\linewidth - 2\ctlpadding)/3}
				\begin{tikzpicture}
					\begin{ctrlaxis}
						\fill[green!25]	(current axis.origin) rectangle (current axis.north east);
						\fill[red!25]	(current axis.origin) rectangle (current axis.north west);
						\fill[green!25]	(current axis.origin) rectangle (current axis.south west);
						\fill[red!25]	(current axis.origin) rectangle (current axis.south east);
					\end{ctrlaxis}
				\end{tikzpicture}
			\end{subfigure}\hfill
			\begin{subfigure}{(\linewidth - 2\ctlpadding)/3}
				\begin{tikzpicture}
					\begin{ctrlaxis}
						% ct
						\addplot[domain=-pi:pi, samples=2, draw=none, fill=green!25] {x*0.019940 + 0.002514} \drawge;
						\addplot[domain=-pi:pi, samples=2, draw=none, fill=red!25] {x*0.019940 + 0.002514} \drawle;
						\addplot[domain=-pi:pi, samples=2] {x*0.019940 + 0.002514};
					\end{ctrlaxis}
				\end{tikzpicture}
			\end{subfigure}\hfill
			\begin{subfigure}{(\linewidth - 2\ctlpadding)/3}
				\centering
				\begin{tikzpicture}
					\begin{ctrlaxis}
						% ct
						\addplot[domain=-pi:pi, samples=2, draw=none, fill=green!25] {x*-0.360049 + 0.021050} \drawle;
						\addplot[domain=-pi:pi, samples=2, draw=none, fill=red!25] {x*-0.360049 + 0.021050} \drawge;
						\addplot[domain=-pi:pi, samples=2] {x*-0.360049 + 0.021050};
					\end{ctrlaxis}
				\end{tikzpicture}
			\end{subfigure}%
			\label{fig:balancing:tt}%
		\end{subfigure}\\
		\vskip \abovecaptionskip
		\begin{subfigure}[t]{\linewidth}
			\setcounter{subfigure}{0}%
			\renewcommand\thesubfigure{\roman{subfigure}}%
			\begin{subfigure}{(\linewidth - 2\ctlpadding)/3}
				\caption{Ideal}
			\end{subfigure}\hfill
			\begin{subfigure}{(\linewidth - 2\ctlpadding)/3}
				\caption{Learnt in simulation}
			\end{subfigure}\hfill
			\begin{subfigure}{(\linewidth - 2\ctlpadding)/3}
				\caption{Learnt in reality}
			\end{subfigure}
		\end{subfigure}
		\caption{Controllers for stabilizing pitch $\psi$ and roll $\theta$}
		\label{fig:balancing}
		\medskip
		\small
		Areas are shaded where \mybox[fill=green!25]{$\tau > 0$} and \mybox[fill=red!25]{$\tau < 0$}.
		$\tau_w > 0$ corresponds to a force driving the robot forwards, and $\tau_t > 0$ corresponds to a moment rotating the robot clockwise
		These plots can be interpreted as a top down view of the robot at the origin facing right, with a point on the plot corresponding to the location of the top of the robot.

	\end{figure}

	\footnotetext{Rotating the turntable counter-clockwise exerts a clockwise moment on the robot body}

\end{document}