\providecommand{\thebibpath}{..}
\makeatletter\def\input@path{{\thebibpath/}{.}}\makeatother
\documentclass[main.tex]{subfiles}
\begin{document}

\subsection{Fixing the cost function}
	\label{sec:cost-function}
	It was discovered that multiple mistakes were present in the cost function. The original cost function was defined as in the documentation as
	\begin{align}
		c(\bm z) = 1 - \exp\left\{-\frac{1}{2}\left(a\|\bm{p} - \bm p_0\|^2+ \frac{\phi^2}{(4\pi)^2}\right)\right\} \,, \label{eq:cost-bad}
	\end{align},
	where $\bm{p} \in \mathbb{R}^3$ is the position of the centre of the turntable in robot-centric coordinates, $\bm{p}_0$ is the target position, and $a$ a dimensionless weight to adjust the relative cost of position vs yaw angle.
	The implementation did not match this, applying different weights to each component of $\bm{p} - \bm{p_0}$, which whilst reasonable, is a form of cost-shaping -- something that should be avoided, as it is providing unnecessary hints to \textsc{Pilco} about how to learn the task.

	Both of these approaches can be shown to be invalid by dimensional analysis -- the addition in \cref{eq:cost-bad} attempts to add a quantity with units \si{\meter\square} to a unitless quantity, which is not a well-defined operation.
	Indeed, this caused noticeable problems when attempting to learn on the small unicycle, as the first term decreased by a factor of $10$ due to the change in size.
	The fix is to form a non-dimensional group with $\bm{p}$ and the robot parameters and use that in place of $\bm{p}$, for which $\frac{\bm{p}}{h}$ is sufficient, giving
	\begin{align}
		c(\bm z) = 1 - \exp\left\{-\frac{1}{2}\left(a\frac{\|\bm{p} - \bm p_0\|}{h}^2+ \frac{\phi^2}{(4\pi)^2}\right)\right\} \,, \label{eq:correct-cost}
	\end{align}
	where $h = r_w + r_f + r_t$ is the total height of the robot.

	There was also a mistake in the algebra for $\|\bm{p} - \bm p_0\|^2$, which omitted a trigonometric term.
	The correct expressions for these vectors, using the states defined in \cref{eq:state-vars} and the robot parameters from \cref{table:mechanical}, is
	\begin{align}
			\bm{p} &= \begin{bmatrix}
				(r_f + r_t) \sin \psi \\
				\big((r_f + r_t) \cos \psi + r_w\big) \sin\theta \\
				\big((r_f + r_t) \cos \psi + r_w\big) \cos\theta
			\end{bmatrix}\,,
		\quad&
			\bm{p}_0 &= \begin{bmatrix}
				x_c \\
				y_c \\
				r_w + r_t + r_f
			\end{bmatrix}\,.
	\end{align}
	The product terms containing $\cos$ and $\sin$ can be combined with
	\begin{align}
		\begin{aligned}
			\cos \psi \sin\theta &= \tfrac{1}{2} \big(
				\sin \beta + \sin \alpha
			\big) \\
			\cos \psi \cos\theta &= \tfrac{1}{2} \big(
				\cos \beta + \cos \alpha
			\big)
		\end{aligned}
		\quad\text{where}\quad
		\begin{aligned}
		\alpha &:= \theta - \psi \\
		\beta &:= \theta + \psi
		\end{aligned}\,,
	\end{align}
	suggesting the augmentation of the state vector with $\sin, \cos$ of $\alpha, \beta$.
	This allows $\bm{p}$ to be expressed as a linear function of the augmented state $\tilde{\bm{z}}$,
	\begin{align}
		\bm{p} &= \underbrace{\begin{bmatrix}
			(r_f + r_t) \unitv{\psi} \\
			\tfrac{1}{2} (r_f + r_t) (\unitv{\sin \beta} + \unitv{\sin \alpha}) +
			 \unitv{\sin \theta} r_w \\
			\tfrac{1}{2} (r_f + r_t) (\unitv{\cos \beta} + \unitv{\cos \alpha}) +
			 \unitv{\cos \theta} r_w
		\end{bmatrix}}_{A_p} \tilde{\bm{z}}, &
		\bm{p}_0 &= \underbrace{\begin{bmatrix}
			\unitv{x_c} \\
			\unitv{y_c} \\
			0
		\end{bmatrix}}_{A_{p_0}} \tilde{\bm{z}} +
		\underbrace{\begin{bmatrix} 0 \\ 0 \\ r_w + r_f + r_t \end{bmatrix}}_{\bm{b}_{p_0}}
	\end{align}
	defining the row vector $\unitv\psi$ such that $\unitv\psi\tilde{\bm{z}} = \psi$, and similarly for all other state variables.
	This lets us express
	\begin{align}
		\|\bm{p} - \bm{p}_0\|^2
			&= \|(A_p - A_{p_0})\tilde{\bm{z}} - \bm{b}_{p_0}\|^2  \nonumber \\
			&= \|(A_p - A_{p_0})(\tilde{\bm{z}} - \tilde{\bm{z}}_{p_0})\|^2 \\
		\text{where}\quad\tilde{\bm{z}}_{p_0}
			&= \unitv{\cos \beta} + \unitv{\cos \alpha} + \unitv{\cos \theta}  \nonumber
	\end{align}
	From here, the cost function in \cref{eq:correct-cost} can be expressed in a simple form,
	\begin{align}
		c(\bm z) &= 1 - \exp\left\{-\frac{1}{2}
			(\tilde{\bm{z}} - \tilde{\bm{z}}_0)^T  Q (\tilde{\bm{z}} - \tilde{\bm{z}}_0)
		\right\} \,, \label{eq:quad-cost} \\
		\text{where}\quad Q &=
			\unitv{\phi}^T\frac{1}{4\pi^2}\unitv{\phi} +
			(A_p - A_{p_0})^T\frac{1}{h^2}(A_p - A_{p_0}) \nonumber \\
			\tilde{\bm{z}}_0 &= \tilde{\bm{z}}_{p_0} \nonumber
	\end{align}

\subsection{Roll limits}
	A suggestion from previous work was to redesign the frame to \enquote{reduce the effect of the roll limitation}\cite[p.~35]{aleksi}, due to a suspicion that the way in which the hardware was limited in roll to \SI{17}{\degree} (due to its axle hitting the ground) was a hindrance to the learning -- indeed, the simulated unicycle did not have this constraint.

	Results from the earlier work were not necessarily valid, as they only evaluated the effect of this yaw limit in simulation on the large unicycle.
	Since this work acquired a model for the small unicycle (\cref{table:mechanical}), a more correct experiment could be performed.
	\Cref{fig:roll:progress:prev} shows the effect of running the algorithm with three different roll limits.
	The difference in performance between \SI{17}{\degree} and \SI{45}{\degree} is minimal, with both showing a dramatic loss of performance compared to the idealized unicycle -- suggesting that a mechanical redesign would not be a good use of time.

	One of the problems encountered due to roll limits is that cost function contains little information about the fact that the axle is about to hit the ground.
	A cost function should penalize undesirable behaviour, so a new term of $\unitv{\theta}\frac{1}{\SI{17}{\degree}}^2\unitv{\theta}^T$ was added to the $Q$ from \cref{eq:quad-cost}.

	\Cref{fig:roll:progress:fixed} shows this gives a dramatic performance increase, with all three configurations now learning a stabilizing controller with similar amounts of experience.
	Of course, the \SI{17}{\degree} configuration requires more iterations, as each is shorter.
	It's clear that there would still be an advantage to a mechanical redesign, but due to its reduced effect, it was decided it was not a good use of time in this work.

	\begin{figure}
	% Gross, but needed to make the `\ref`s work below
		\definecolor{mycolor1}{rgb}{0.00000,0.44700,0.74100}%
		\definecolor{mycolor2}{rgb}{0.85000,0.32500,0.09800}%
		\definecolor{mycolor3}{rgb}{0.92900,0.69400,0.12500}%
		\begin{subfigure}[t]{\linewidth}
			\begin{minipage}{0.5\linewidth - 2em}
				% This file was created by matlab2tikz.
%
%The latest updates can be retrieved from
%  http://www.mathworks.com/matlabcentral/fileexchange/22022-matlab2tikz-matlab2tikz
%where you can also make suggestions and rate matlab2tikz.
%
\definecolor{mycolor1}{rgb}{0.00000,0.44700,0.74100}%
\definecolor{mycolor2}{rgb}{0.85000,0.32500,0.09800}%
\definecolor{mycolor3}{rgb}{0.92900,0.69400,0.12500}%
%
\begin{tikzpicture}[%
trim axis left,
trim axis right
]

\begin{axis}[%
width=\linewidth,
height=3cm,
at={(0\linewidth,0cm)},
scale only axis,
xmin=0.5,
xmax=50.5,
xlabel style={font=\color{white!15!black}},
xlabel={iteration number},
ymin=-0.1,
ymax=2.6,
ylabel style={font=\color{white!15!black}},
ylabel={time upright / $\si{\second}$},
axis background/.style={fill=white},
axis x line*=bottom,
axis y line*=left,
legend style={at={(0.03,0.97)}, anchor=north west, legend cell align=left, align=left, draw=white!15!black}
]
\addplot [color=mycolor1]
  table[row sep=crcr]{%
1	0.15\\
2	0.25\\
3	0.1\\
4	0.1\\
5	0.15\\
6	0.2\\
7	0.1\\
8	0.15\\
9	0.15\\
10	0.15\\
11	0.35\\
12	0.35\\
13	0.8\\
14	0.6\\
15	0.35\\
16	0.55\\
17	0.35\\
18	0.7\\
19	0.45\\
20	0.5\\
21	0.6\\
22	0.75\\
23	1.35\\
24	0.75\\
25	1.8\\
26	0.7\\
27	2.1\\
28	2.05\\
29	2.5\\
30	2.5\\
31	2.5\\
32	2.5\\
33	2.3\\
34	2.15\\
35	2.5\\
36	2.5\\
37	2.5\\
38	2.5\\
39	2.5\\
40	2.5\\
41	2.5\\
42	2.5\\
43	2.5\\
44	2.5\\
45	2.5\\
46	2.5\\
47	1.9\\
48	2.5\\
49	2.5\\
50	2.5\\
}; \label{line:sim:90};
\addlegendentry{\SI{90}{\degree}}

\addplot [color=mycolor2]
  table[row sep=crcr]{%
1	0.15\\
2	0.25\\
3	0.1\\
4	0.1\\
5	0.15\\
6	0.2\\
7	0.1\\
8	0.15\\
9	0.15\\
10	0.15\\
11	0.35\\
12	0.45\\
13	0.25\\
14	0.25\\
15	0.45\\
16	0.45\\
17	0.3\\
18	0.3\\
19	0.45\\
20	0.55\\
21	0.45\\
22	0.45\\
23	0.6\\
24	0.6\\
25	0.5\\
26	0.6\\
27	0.75\\
28	0.5\\
29	0.45\\
30	0.55\\
31	0.5\\
32	0.55\\
33	0.6\\
34	1.15\\
35	0.8\\
36	0.55\\
37	0.6\\
38	0.55\\
39	0.4\\
40	0.55\\
41	0.75\\
42	0.7\\
43	0.45\\
44	0.5\\
45	0.85\\
46	0.8\\
47	1.25\\
48	0.65\\
49	1.45\\
50	0.85\\
}; \label{line:sim:45};
\addlegendentry{\SI{45}{\degree}}

\addplot [color=mycolor3]
  table[row sep=crcr]{%
1	0.15\\
2	0.2\\
3	0.15\\
4	0.1\\
5	0.15\\
6	0.15\\
7	0.2\\
8	0.15\\
9	0.15\\
10	0.15\\
11	0.15\\
12	0.15\\
13	0.45\\
14	0.2\\
15	0.6\\
16	0.25\\
17	0.65\\
18	0.15\\
19	0.3\\
20	0.45\\
21	0.45\\
22	0.45\\
23	0.55\\
24	0.55\\
25	0.3\\
26	0.45\\
27	0.45\\
28	0.4\\
29	0.45\\
30	0.3\\
31	0.55\\
32	0.4\\
33	0.4\\
34	0.35\\
35	0.25\\
36	0.6\\
37	0.55\\
38	0.35\\
39	0.6\\
40	0.55\\
41	0.5\\
42	0.5\\
43	0.6\\
44	0.5\\
45	0.5\\
46	0.5\\
47	0.35\\
48	0.55\\
49	0.3\\
50	0.55\\
}; \label{line:sim:17};
\addlegendentry{\SI{17}{\degree}}

\end{axis}
\end{tikzpicture}%
			\end{minipage}\hfill
			\begin{minipage}{0.5\linewidth - 2em}
				% This file was created by matlab2tikz.
%
%The latest updates can be retrieved from
%  http://www.mathworks.com/matlabcentral/fileexchange/22022-matlab2tikz-matlab2tikz
%where you can also make suggestions and rate matlab2tikz.
%
\definecolor{mycolor1}{rgb}{0.00000,0.44700,0.74100}%
\definecolor{mycolor2}{rgb}{0.85000,0.32500,0.09800}%
\definecolor{mycolor3}{rgb}{0.92900,0.69400,0.12500}%
%
\begin{tikzpicture}[%
trim axis left,
trim axis right
]

\begin{axis}[%
width=\linewidth,
height=3cm,
at={(0\linewidth,0cm)},
scale only axis,
xmin=0,
xmax=40,
xlabel style={font=\color{white!15!black}},
xlabel={total experience / $\si{\second}$},
ymin=-0.1,
ymax=2.6,
ylabel style={font=\color{white!15!black}},
ylabel={time upright / $\si{\second}$},
axis background/.style={fill=white},
axis x line*=bottom,
axis y line*=left
]
\addplot [color=mycolor1]
  table[row sep=crcr]{%
0.15	0.15\\
0.4	0.25\\
0.5	0.1\\
0.6	0.1\\
0.75	0.15\\
0.95	0.2\\
1.05	0.1\\
1.2	0.15\\
1.35	0.15\\
1.5	0.15\\
1.85	0.35\\
2.2	0.35\\
3	0.8\\
3.6	0.6\\
3.95	0.35\\
4.5	0.55\\
4.85	0.35\\
5.55	0.7\\
6	0.45\\
6.5	0.5\\
7.1	0.6\\
7.85	0.75\\
9.2	1.35\\
9.95	0.75\\
11.75	1.8\\
12.45	0.7\\
14.55	2.1\\
16.6	2.05\\
19.1	2.5\\
21.6	2.5\\
24.1	2.5\\
26.6	2.5\\
28.9	2.3\\
31.05	2.15\\
33.55	2.5\\
36.05	2.5\\
38.55	2.5\\
41.05	2.5\\
43.55	2.5\\
46.05	2.5\\
48.55	2.5\\
51.05	2.5\\
53.55	2.5\\
56.05	2.5\\
58.55	2.5\\
61.05	2.5\\
62.95	1.9\\
65.45	2.5\\
67.95	2.5\\
70.45	2.5\\
};
\addlegendentry{\SI{90}{\degree}}

\addplot [color=mycolor2]
  table[row sep=crcr]{%
0.15	0.15\\
0.4	0.25\\
0.5	0.1\\
0.6	0.1\\
0.75	0.15\\
0.95	0.2\\
1.05	0.1\\
1.2	0.15\\
1.35	0.15\\
1.5	0.15\\
1.85	0.35\\
2.3	0.45\\
2.55	0.25\\
2.8	0.25\\
3.25	0.45\\
3.7	0.45\\
4	0.3\\
4.3	0.3\\
4.75	0.45\\
5.3	0.55\\
5.75	0.45\\
6.2	0.45\\
6.8	0.6\\
7.4	0.6\\
7.9	0.5\\
8.5	0.6\\
9.25	0.75\\
9.75	0.5\\
10.2	0.45\\
10.75	0.55\\
11.25	0.5\\
11.8	0.55\\
12.4	0.6\\
13.55	1.15\\
14.35	0.8\\
14.9	0.55\\
15.5	0.6\\
16.05	0.55\\
16.45	0.4\\
17	0.55\\
17.75	0.75\\
18.45	0.7\\
18.9	0.45\\
19.4	0.5\\
20.25	0.85\\
21.05	0.8\\
22.3	1.25\\
22.95	0.65\\
24.4	1.45\\
25.25	0.85\\
};
\addlegendentry{\SI{45}{\degree}}

\addplot [color=mycolor3]
  table[row sep=crcr]{%
0.15	0.15\\
0.35	0.2\\
0.5	0.15\\
0.6	0.1\\
0.75	0.15\\
0.9	0.15\\
1.1	0.2\\
1.25	0.15\\
1.4	0.15\\
1.55	0.15\\
1.7	0.15\\
1.85	0.15\\
2.3	0.45\\
2.5	0.2\\
3.1	0.6\\
3.35	0.25\\
4	0.65\\
4.15	0.15\\
4.45	0.3\\
4.9	0.45\\
5.35	0.45\\
5.8	0.45\\
6.35	0.55\\
6.9	0.55\\
7.2	0.3\\
7.65	0.45\\
8.1	0.45\\
8.5	0.4\\
8.95	0.45\\
9.25	0.3\\
9.8	0.55\\
10.2	0.4\\
10.6	0.4\\
10.95	0.35\\
11.2	0.25\\
11.8	0.6\\
12.35	0.55\\
12.7	0.35\\
13.3	0.6\\
13.85	0.55\\
14.35	0.5\\
14.85	0.5\\
15.45	0.6\\
15.95	0.5\\
16.45	0.5\\
16.95	0.5\\
17.3	0.35\\
17.85	0.55\\
18.15	0.3\\
18.7	0.55\\
};
\addlegendentry{\SI{17}{\degree}}
\legend{};

\end{axis}
\end{tikzpicture}%
			\end{minipage}
			\caption{With corrected cost function from \cref{sec:cost-function}}
			\label{fig:roll:progress:prev}
		\end{subfigure}
		\par\bigskip
		\begin{subfigure}[t]{\linewidth}
			\begin{minipage}{0.5\linewidth - 2em}
				% This file was created by matlab2tikz.
%
%The latest updates can be retrieved from
%  http://www.mathworks.com/matlabcentral/fileexchange/22022-matlab2tikz-matlab2tikz
%where you can also make suggestions and rate matlab2tikz.
%
\definecolor{mycolor1}{rgb}{0.00000,0.44700,0.74100}%
\definecolor{mycolor2}{rgb}{0.85000,0.32500,0.09800}%
\definecolor{mycolor3}{rgb}{0.92900,0.69400,0.12500}%
%
\begin{tikzpicture}[%
trim axis left,
trim axis right
]

\begin{axis}[%
width=0.951\linewidth,
height=3cm,
at={(0\linewidth,0cm)},
scale only axis,
xmin=0.5,
xmax=50.5,
xlabel style={font=\color{white!15!black}},
xlabel={iteration number},
ymin=-0.1,
ymax=2.6,
ylabel style={font=\color{white!15!black}},
ylabel={time upright / $\si{\second}$},
axis background/.style={fill=white},
axis x line*=bottom,
axis y line*=left,
legend style={at={(0.03,0.97)}, anchor=north west, legend cell align=left, align=left, draw=white!15!black}
]
\addplot [color=mycolor1]
  table[row sep=crcr]{%
1	0.15\\
2	0.25\\
3	0.1\\
4	0.1\\
5	0.15\\
6	0.2\\
7	0.1\\
8	0.15\\
9	0.15\\
10	0.15\\
11	0.4\\
12	0.35\\
13	0.55\\
14	0.5\\
15	0.6\\
16	0.3\\
17	0.3\\
18	0.25\\
19	0.25\\
20	0.3\\
21	0.35\\
22	0.3\\
23	0.5\\
24	0.4\\
25	0.85\\
26	0.5\\
27	0.5\\
28	1.4\\
29	0.95\\
30	1.1\\
31	1.5\\
32	2.25\\
33	2.1\\
34	2.2\\
35	2\\
36	1.5\\
37	2.5\\
38	2.5\\
39	1.7\\
40	2.5\\
41	2.5\\
42	2.5\\
43	2.5\\
44	2.5\\
45	2.5\\
46	2.5\\
47	2.5\\
48	2.5\\
49	2.5\\
50	2.5\\
};
\addlegendentry{\SI{90}{\degree}}

\addplot [color=mycolor2]
  table[row sep=crcr]{%
1	0.15\\
2	0.25\\
3	0.1\\
4	0.1\\
5	0.15\\
6	0.2\\
7	0.1\\
8	0.15\\
9	0.15\\
10	0.15\\
11	0.5\\
12	0.35\\
13	0.2\\
14	0.3\\
15	0.2\\
16	0.35\\
17	0.3\\
18	0.45\\
19	0.45\\
20	0.45\\
21	0.6\\
22	0.45\\
23	0.55\\
24	0.55\\
25	0.7\\
26	0.45\\
27	0.55\\
28	0.9\\
29	0.75\\
30	0.6\\
31	1\\
32	1.1\\
33	1.6\\
34	1.65\\
35	1.45\\
36	1.8\\
37	2.05\\
38	1.45\\
39	2.5\\
40	2.5\\
41	2.1\\
42	2.5\\
43	2.5\\
44	2.5\\
45	2.5\\
46	2.5\\
47	2.4\\
48	2.5\\
49	2.35\\
50	2.5\\
};
\addlegendentry{\SI{45}{\degree}}

\addplot [color=mycolor3]
  table[row sep=crcr]{%
1	0.15\\
2	0.2\\
3	0.15\\
4	0.1\\
5	0.15\\
6	0.15\\
7	0.2\\
8	0.15\\
9	0.15\\
10	0.15\\
11	0.2\\
12	0.25\\
13	0.25\\
14	0.35\\
15	0.25\\
16	0.25\\
17	0.3\\
18	0.35\\
19	0.4\\
20	0.4\\
21	0.3\\
22	0.4\\
23	0.35\\
24	0.3\\
25	0.65\\
26	1.05\\
27	0.65\\
28	0.5\\
29	0.5\\
30	0.75\\
31	0.8\\
32	0.95\\
33	0.5\\
34	0.85\\
35	1.2\\
36	1\\
37	0.95\\
38	1.65\\
39	1.55\\
40	1.65\\
41	1.6\\
42	2.45\\
43	1.4\\
44	1.65\\
45	2.2\\
46	1.35\\
47	2.5\\
48	1.7\\
49	1.6\\
50	1.25\\
};
\addlegendentry{\SI{17}{\degree}}

\end{axis}
\end{tikzpicture}%
			\end{minipage}\hfill
			\begin{minipage}{0.5\linewidth - 2em}
				% This file was created by matlab2tikz.
%
%The latest updates can be retrieved from
%  http://www.mathworks.com/matlabcentral/fileexchange/22022-matlab2tikz-matlab2tikz
%where you can also make suggestions and rate matlab2tikz.
%
\definecolor{mycolor1}{rgb}{0.00000,0.44700,0.74100}%
\definecolor{mycolor2}{rgb}{0.85000,0.32500,0.09800}%
\definecolor{mycolor3}{rgb}{0.92900,0.69400,0.12500}%
%
\begin{tikzpicture}[%
trim axis left,
trim axis right
]

\begin{axis}[%
width=0.951\linewidth,
height=3cm,
at={(0\linewidth,0cm)},
scale only axis,
xmin=0,
xmax=40,
xlabel style={font=\color{white!15!black}},
xlabel={total experience / $\si{\second}$},
ymin=-0.1,
ymax=2.6,
ylabel style={font=\color{white!15!black}},
ylabel={time upright / $\si{\second}$},
axis background/.style={fill=white},
axis x line*=bottom,
axis y line*=left,
legend style={at={(0.03,0.97)}, anchor=north west, legend cell align=left, align=left, draw=white!15!black}
]
\addplot [color=mycolor1]
  table[row sep=crcr]{%
0.15	0.15\\
0.4	0.25\\
0.5	0.1\\
0.6	0.1\\
0.75	0.15\\
0.95	0.2\\
1.05	0.1\\
1.2	0.15\\
1.35	0.15\\
1.5	0.15\\
1.9	0.4\\
2.25	0.35\\
2.8	0.55\\
3.3	0.5\\
3.9	0.6\\
4.2	0.3\\
4.5	0.3\\
4.75	0.25\\
5	0.25\\
5.3	0.3\\
5.65	0.35\\
5.95	0.3\\
6.45	0.5\\
6.85	0.4\\
7.7	0.85\\
8.2	0.5\\
8.7	0.5\\
10.1	1.4\\
11.05	0.95\\
12.15	1.1\\
13.65	1.5\\
15.9	2.25\\
18	2.1\\
20.2	2.2\\
22.2	2\\
23.7	1.5\\
26.2	2.5\\
28.7	2.5\\
30.4	1.7\\
32.9	2.5\\
35.4	2.5\\
37.9	2.5\\
40.4	2.5\\
42.9	2.5\\
45.4	2.5\\
47.9	2.5\\
50.4	2.5\\
52.9	2.5\\
55.4	2.5\\
57.9	2.5\\
};
\addlegendentry{\SI{90}{\degree}}

\addplot [color=mycolor2]
  table[row sep=crcr]{%
0.15	0.15\\
0.4	0.25\\
0.5	0.1\\
0.6	0.1\\
0.75	0.15\\
0.95	0.2\\
1.05	0.1\\
1.2	0.15\\
1.35	0.15\\
1.5	0.15\\
2	0.5\\
2.35	0.35\\
2.55	0.2\\
2.85	0.3\\
3.05	0.2\\
3.4	0.35\\
3.7	0.3\\
4.15	0.45\\
4.6	0.45\\
5.05	0.45\\
5.65	0.6\\
6.1	0.45\\
6.65	0.55\\
7.2	0.55\\
7.9	0.7\\
8.35	0.45\\
8.9	0.55\\
9.8	0.9\\
10.55	0.75\\
11.15	0.6\\
12.15	1\\
13.25	1.1\\
14.85	1.6\\
16.5	1.65\\
17.95	1.45\\
19.75	1.8\\
21.8	2.05\\
23.25	1.45\\
25.75	2.5\\
28.25	2.5\\
30.35	2.1\\
32.85	2.5\\
35.35	2.5\\
37.85	2.5\\
40.35	2.5\\
42.85	2.5\\
45.25	2.4\\
47.75	2.5\\
50.1	2.35\\
52.6	2.5\\
};
\addlegendentry{\SI{45}{\degree}}

\addplot [color=mycolor3]
  table[row sep=crcr]{%
0.15	0.15\\
0.35	0.2\\
0.5	0.15\\
0.6	0.1\\
0.75	0.15\\
0.9	0.15\\
1.1	0.2\\
1.25	0.15\\
1.4	0.15\\
1.55	0.15\\
1.75	0.2\\
2	0.25\\
2.25	0.25\\
2.6	0.35\\
2.85	0.25\\
3.1	0.25\\
3.4	0.3\\
3.75	0.35\\
4.15	0.4\\
4.55	0.4\\
4.85	0.3\\
5.25	0.4\\
5.6	0.35\\
5.9	0.3\\
6.55	0.65\\
7.6	1.05\\
8.25	0.65\\
8.75	0.5\\
9.25	0.5\\
10	0.75\\
10.8	0.8\\
11.75	0.95\\
12.25	0.5\\
13.1	0.85\\
14.3	1.2\\
15.3	1\\
16.25	0.95\\
17.9	1.65\\
19.45	1.55\\
21.1	1.65\\
22.7	1.6\\
25.15	2.45\\
26.55	1.4\\
28.2	1.65\\
30.4	2.2\\
31.75	1.35\\
34.25	2.5\\
35.95	1.7\\
37.55	1.6\\
38.8	1.25\\
};
\addlegendentry{\SI{17}{\degree}}

\end{axis}
\end{tikzpicture}%
			\end{minipage}
			\caption{As above, with added cost penalty for roll}
			\label{fig:roll:progress:fixed}
		\end{subfigure}
		\caption{Effect of roll-limit on learning progress}
		\label{fig:roll:progress}
		\medskip
		\small
		Showing the idealized limit of \SI{90}{\degree} (\ref{line:sim:90}), the actual value of \SI{17}{\degree} (\ref{line:sim:17}), and an optimistic value achievable by a mechanical redesign of \SI{45}{\degree} (\ref{line:sim:45}).
		\medskip\\
		This data is taken from just one learning run, since the duration of such a run prohibits many repeat trials, ranging from \SI{3.5}{\hour} for \SI{17}{\degree} to \SI{11}{\hour} for \SI{90}{\degree}.
		Despite the random number generated being seeded identically for all 3 cases, the initial robot states will differ, due to a different volume of consumption of the random number stream whilst using the random controller.
	\end{figure}

	\bib

\end{document}
