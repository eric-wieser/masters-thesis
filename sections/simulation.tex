\providecommand{\thebibpath}{..}
\makeatletter\def\input@path{{\thebibpath/}{.}}\makeatother
\documentclass[main.tex]{subfiles}
\begin{document}

\subsection{Fixing the cost function}
	\label{sec:cost-function}
	It was discovered that multiple mistakes were present in the cost function. The original cost function was defined as in the documentation as
	\begin{align}
		c(\bm z) = 1 - \exp\left\{-\frac{1}{2}\left(a\|\bm{p} - \bm p_0\|^2+ \frac{\phi^2}{(4\pi)^2}\right)\right\} \,, \label{eq:cost-bad}
	\end{align},
	where $\bm{p} \in \mathbb{R}^3$ is the position of the centre of the turntable in robot-centric coordinates, $\bm{p}_0$ is the target position, and $a$ a dimensionless weight to adjust the relative cost of position vs yaw angle.
	The implementation did not match this, applying different weights to each component of $\bm{p} - \bm{p_0}$, which whilst reasonable, is a form of cost-shaping -- something that should be avoided, as it is providing unnecessary hints to \textsc{Pilco} about how to learn the task.

	Both of these approaches can be shown to be invalid by dimensional analysis -- the addition in \cref{eq:cost-bad} attempts to add a quantity with units \si{\meter\square} to a unitless quantity, which is not a well-defined operation.
	Indeed, this caused noticeable problems when attempting to learn on the small unicycle, as the first term decreased by a factor of $10$ due to the change in size.
	The fix is to form a non-dimensional group with $\bm{p}$ and the robot parameters and use that in place of $\bm{p}$, for which $\frac{\bm{p}}{h}$ is sufficient, giving
	\begin{align}
		c(\bm z) = 1 - \exp\left\{-\frac{1}{2}\left(a\frac{\|\bm{p} - \bm p_0\|}{h}^2+ \frac{\phi^2}{(4\pi)^2}\right)\right\} \,, \label{eq:correct-cost}
	\end{align}
	where $h = r_w + r_f + r_t$ is the total height of the robot.

	There was also a mistake in the algebra for $\|\bm{p} - \bm p_0\|^2$, which omitted a trigonometric term.
	The correct expressions for these vectors, using the states defined in \cref{eq:state-vars} and the robot parameters from \cref{table:mechanical}, is
	\begin{align}
			\bm{p} &= \begin{bmatrix}
				(r_f + r_t) \sin \psi \\
				\big((r_f + r_t) \cos \psi + r_w\big) \sin\theta \\
				\big((r_f + r_t) \cos \psi + r_w\big) \cos\theta
			\end{bmatrix}\,,
		\quad&
			\bm{p}_0 &= \begin{bmatrix}
				x_c \\
				y_c \\
				r_w + r_t + r_f
			\end{bmatrix}\,.
	\end{align}
	The product terms containing $\cos$ and $\sin$ can be combined with
	\begin{align}
		\begin{aligned}
			\cos \psi \sin\theta &= \tfrac{1}{2} \big(
				\sin \beta + \sin \alpha
			\big) \\
			\cos \psi \cos\theta &= \tfrac{1}{2} \big(
				\cos \beta + \cos \alpha
			\big)
		\end{aligned}
		\quad\text{where}\quad
		\begin{aligned}
		\alpha &:= \theta - \psi \\
		\beta &:= \theta + \psi
		\end{aligned}\,,
	\end{align}
	suggesting the augmentation of the state vector with $\sin, \cos$ of $\alpha, \beta$.
	This allows $\bm{p}$ to be expressed as a linear function of the augmented state $\tilde{\bm{z}}$,
	\begin{align}
		\bm{p} &= \underbrace{\begin{bmatrix}
			(r_f + r_t) \unitv{\psi} \\
			\tfrac{1}{2} (r_f + r_t) (\unitv{\sin \beta} + \unitv{\sin \alpha}) +
			 \unitv{\sin \theta} r_w \\
			\tfrac{1}{2} (r_f + r_t) (\unitv{\cos \beta} + \unitv{\cos \alpha}) +
			 \unitv{\cos \theta} r_w
		\end{bmatrix}}_{A_p} \tilde{\bm{z}}, &
		\bm{p}_0 &= \underbrace{\begin{bmatrix}
			\unitv{x_c} \\
			\unitv{y_c} \\
			0
		\end{bmatrix}}_{A_{p_0}} \tilde{\bm{z}} +
		\underbrace{\begin{bmatrix} 0 \\ 0 \\ r_w + r_f + r_t \end{bmatrix}}_{\bm{b}_{p_0}}
	\end{align}
	defining the row vector $\unitv\psi$ such that $\unitv\psi\tilde{\bm{z}} = \psi$, and similarly for all other state variables.
	This lets us express
	\begin{align}
		\|\bm{p} - \bm{p}_0\|^2
			&= \|(A_p - A_{p_0})\tilde{\bm{z}} - \bm{b}_{p_0}\|^2  \nonumber \\
			&= \|(A_p - A_{p_0})(\tilde{\bm{z}} - \tilde{\bm{z}}_{p_0})\|^2 \\
		\text{where}\quad\tilde{\bm{z}}_{p_0}
			&= \unitv{\cos \beta} + \unitv{\cos \alpha} + \unitv{\cos \theta}  \nonumber
	\end{align}
	From here, the cost function in \cref{eq:correct-cost} can be expressed in a simple form,
	\begin{align}
		c(\bm z) &= 1 - \exp\left\{-\frac{1}{2}
			(\tilde{\bm{z}} - \tilde{\bm{z}}_0)^T  Q (\tilde{\bm{z}} - \tilde{\bm{z}}_0)
		\right\} \,, \label{eq:quad-cost} \\
		\text{where}\quad Q &=
			\unitv{\phi}^T\frac{1}{4\pi^2}\unitv{\phi} +
			(A_p - A_{p_0})^T\frac{1}{h^2}(A_p - A_{p_0}) \nonumber \\
			\tilde{\bm{z}}_0 &= \tilde{\bm{z}}_{p_0} \nonumber
	\end{align}

\subsection{Roll limits}
	A suggestion from previous work was to redesign the frame to \enquote{reduce the effect of the roll limitation}\cite[p.~35]{aleksi}, due to a suspicion that the way in which the hardware was limited in roll to \SI{17}{\degree} was a hindrance to the learning -- indeed, the simulated unicycle did not have this constraint.

	Results from the earlier work were not necessarily valid, as they only evaluated the effect of this yaw limit in simulation on the large unicycle.
	Since this work acquired a model for the small unicycle (\cref{table:mechanical}), a more correct experiment could be performed.

	The results did indeed show a reduced ability to learn. \todo{Plots to show this}
	However, it was realized that some cost-shaping would become desirable - there should be a penalty for coming close to the roll constraint, and so a new term was added to $Q$ from \cref{eq:quad-cost}, of $\unitv{\theta}\frac{1}{\SI{17}{\degree}}^2\unitv{\theta}^T$.
	This did indeed improve performance \todo{Plots to show this}.
\end{document}
