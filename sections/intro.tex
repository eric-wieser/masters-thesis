\providecommand{\thebibpath}{..}
\makeatletter\def\input@path{{\thebibpath/}{.}}\makeatother
\documentclass[main.tex]{subfiles}
\begin{document}

\section{Optimal Control of Discrete systems}
	One way to model dynamical systems is with discrete-time equations, which for a sufficiently small timestep, are close approximations to their continuous-time counterparts.
	For a suitable choice of state vector $\bm{x}$ and timestep $\delta t$, discrete-time systems can be described in terms of the transition dynamics,
	\begin{align}
		\bm{x}\tind{i + 1} = f(\bm{x}\tind{i}, \bm{u}\tind{i}) \quad \text{where} \quad \bm{x}\tind{i} := x(t_0 + i\cdot\delta t)\,. \label{eq:transition}
	\end{align}
	From this, the optimal control problem in discrete time can be stated as
	\begin{alignat}{2}
		\text{find}&& \quad
			\pi^*(\bm{x}) &= \argmin_{\pi(\bm{x})}
				J(\bm{x}\tind\cdot, \bm{u}\tind\cdot)
				\quad \text{where} \quad
				J(\bm{x}\tind\cdot, \bm{u}\tind\cdot) = \sum_{i=0}^{i=N} c(\bm{x}\tind{i}, \bm{u}\tind{i})\quad \\ \label{eq:optimal}
		\text{st.}&& \quad
			\bm{x}\tind{i+1} &= f(\bm{x}\tind{i}, \bm{u}\tind{i}) \\ \nonumber
		&&
			\bm{u}\tind{i} &= \pi(\bm{x}\tind{i})\, \nonumber
	\end{alignat}
	where $\pi(\bm{x})$ is a policy that computes the desired actions for a given state, and $c(\bm{x}, \bm{u})$ attributes a cost to a set of states and actions at a single timestep.

\section{Gaussian Processes}

	\begin{figure}[b]
		\centering
		\begin{subfigure}[t]{0.3\linewidth}
			% This file was created by matlab2tikz.
%
%The latest updates can be retrieved from
%  http://www.mathworks.com/matlabcentral/fileexchange/22022-matlab2tikz-matlab2tikz
%where you can also make suggestions and rate matlab2tikz.
%
\definecolor{mycolor1}{rgb}{0.63500,0.07800,0.18400}%
%
\begin{tikzpicture}[%
trim axis left,
trim axis right
]

\begin{axis}[%
width=0.951\linewidth,
height=3cm,
at={(0\linewidth,0cm)},
scale only axis,
xmin=0,
xmax=4,
xtick={0,0.5,1,1.5,2,2.5,3,3.5,4},
xticklabels={{}},
xlabel style={font=\color{white!15!black}},
xlabel={$x$},
ymin=0,
ymax=0.8,
ytick={0,0.1,0.2,0.3,0.4,0.5,0.6,0.7,0.8},
yticklabels={{}},
ylabel style={font=\color{white!15!black}},
ylabel={$p(x; \mu, \sigma)$},
axis background/.style={fill=white},
axis x line*=bottom,
axis y line*=left
]

\addplot[area legend, draw=none, fill=gray, fill opacity=0.5, forget plot]
table[row sep=crcr] {%
x	y\\
1.17757318652426	0\\
1.17757318652426	0.206271280750742\\
1.19418786962478	0.217739097897359\\
1.2108025527253	0.229590825441285\\
1.22741723582582	0.241820489345824\\
1.24403191892634	0.254420509452383\\
1.26064660202686	0.267381648294113\\
1.27726128512738	0.280692966593307\\
1.2938759682279	0.294341786092705\\
1.31049065132842	0.308313660346281\\
1.32710533442894	0.322592354063595\\
1.34372001752946	0.337159831563359\\
1.36033470062998	0.351996254846595\\
1.3769493837305	0.367079991747673\\
1.39356406683102	0.382387634562911\\
1.41017874993154	0.397894029491587\\
1.42679343303206	0.413572317153464\\
1.44340811613258	0.429393984370811\\
1.4600227992331	0.445328927321905\\
1.47663748233362	0.461345526087756\\
1.49325216543414	0.47741073052501\\
1.50986684853466	0.493490157306383\\
1.52648153163518	0.509548197876469\\
1.5430962147357	0.52554813697604\\
1.55971089783622	0.541452281293194\\
1.57632558093674	0.557222097705589\\
1.59294026403726	0.572818360485709\\
1.60955494713778	0.588201306751485\\
1.6261696302383	0.603330799358753\\
1.64278431333882	0.618166496350794\\
1.65939899643934	0.632668026004713\\
1.67601367953986	0.646795166445348\\
1.69262836264038	0.660508028735915\\
1.7092430457409	0.673767242301157\\
1.72585772884142	0.686534141494462\\
1.74247241194194	0.69877095208558\\
1.75908709504246	0.710440976420969\\
1.77570177814298	0.721508775994845\\
1.7923164612435	0.731940350165936\\
1.80893114434402	0.74170330976318\\
1.82554582744454	0.750767044343062\\
1.84216051054506	0.759102881892104\\
1.85877519364558	0.766684239809985\\
1.8753898767461	0.773486766061606\\
1.89200455984662	0.779488469449809\\
1.90861924294714	0.784669838033812\\
1.92523392604766	0.789013944801169\\
1.94184860914818	0.792506539792436\\
1.9584632922487	0.795136127976925\\
1.97507797534922	0.796894032283945\\
1.99169265844974	0.797774441305843\\
2.00830734155026	0.797774441305843\\
2.02492202465078	0.796894032283945\\
2.0415367077513	0.795136127976925\\
2.05815139085182	0.792506539792436\\
2.07476607395234	0.789013944801169\\
2.09138075705286	0.784669838033812\\
2.10799544015338	0.779488469449809\\
2.1246101232539	0.773486766061606\\
2.14122480635442	0.766684239809985\\
2.15783948945494	0.759102881892104\\
2.17445417255546	0.750767044343062\\
2.19106885565598	0.74170330976318\\
2.2076835387565	0.731940350165936\\
2.22429822185702	0.721508775994845\\
2.24091290495754	0.710440976420969\\
2.25752758805806	0.69877095208558\\
2.27414227115858	0.686534141494462\\
2.2907569542591	0.673767242301157\\
2.30737163735962	0.660508028735915\\
2.32398632046014	0.646795166445348\\
2.34060100356066	0.632668026004713\\
2.35721568666118	0.618166496350794\\
2.3738303697617	0.603330799358753\\
2.39044505286222	0.588201306751485\\
2.40705973596274	0.572818360485709\\
2.42367441906326	0.557222097705589\\
2.44028910216378	0.541452281293194\\
2.4569037852643	0.52554813697604\\
2.47351846836482	0.509548197876469\\
2.49013315146534	0.493490157306383\\
2.50674783456586	0.47741073052501\\
2.52336251766638	0.461345526087757\\
2.5399772007669	0.445328927321905\\
2.55659188386742	0.429393984370811\\
2.57320656696794	0.413572317153464\\
2.58982125006846	0.397894029491587\\
2.60643593316898	0.382387634562911\\
2.6230506162695	0.367079991747673\\
2.63966529937002	0.351996254846596\\
2.65627998247054	0.337159831563359\\
2.67289466557106	0.322592354063595\\
2.68950934867158	0.308313660346281\\
2.7061240317721	0.294341786092705\\
2.72273871487262	0.280692966593307\\
2.73935339797314	0.267381648294114\\
2.75596808107366	0.254420509452383\\
2.77258276417418	0.241820489345824\\
2.7891974472747	0.229590825441285\\
2.80581213037522	0.217739097897359\\
2.82242681347574	0.206271280750742\\
2.82242681347574	0\\
}--cycle;
\addplot [color=black, forget plot]
  table[row sep=crcr]{%
0	0.000267660451529771\\
0.0404040404040404	0.000368590604636303\\
0.0808080808080808	0.000504276112305317\\
0.121212121212121	0.000685419745917184\\
0.161616161616162	0.0009255692289194\\
0.202020202020202	0.00124172459833281\\
0.242424242424242	0.00165502950936388\\
0.282828282828283	0.00219154442862412\\
0.323232323232323	0.00288309463307419\\
0.363636363636364	0.00376817962030763\\
0.404040404040404	0.00489292293663676\\
0.444444444444444	0.00631203263283609\\
0.484848484848485	0.00808973277177296\\
0.525252525252525	0.0103006159847218\\
0.565656565656566	0.0130303565045358\\
0.606060606060606	0.0163762130535745\\
0.646464646464647	0.0204472422439212\\
0.686868686868687	0.0253641366983196\\
0.727272727272727	0.0312585989537112\\
0.767676767676768	0.0382721634279922\\
0.808080808080808	0.0465543853321692\\
0.848484848484849	0.0562603282745569\\
0.888888888888889	0.0675473020705413\\
0.929292929292929	0.0805708292326468\\
0.96969696969697	0.0954798526137071\\
1.01010101010101	0.112411237017888\\
1.05050505050505	0.131483662992913\\
1.09090909090909	0.152791059570133\\
1.13131313131313	0.176395771937897\\
1.17171717171717	0.202321706924249\\
1.21212121212121	0.230547740368852\\
1.25252525252525	0.261001702453701\\
1.29292929292929	0.293555276383304\\
1.33333333333333	0.328020149351987\\
1.37373737373737	0.364145740040454\\
1.41414141414141	0.401618792392587\\
1.45454545454545	0.440065070739981\\
1.4949494949495	0.47905331740255\\
1.53535353535354	0.518101543059398\\
1.57575757575758	0.556685616223426\\
1.61616161616162	0.594250006109949\\
1.65656565656566	0.63022041913464\\
1.6969696969697	0.664017959950017\\
1.73737373737374	0.695074350302397\\
1.77777777777778	0.722847659765487\\
1.81818181818182	0.746837947507958\\
1.85858585858586	0.76660218834497\\
1.8989898989899	0.781767862399903\\
1.93939393939394	0.792044626781266\\
1.97979797979798	0.797233558647621\\
2.02020202020202	0.797233558647621\\
2.06060606060606	0.792044626781266\\
2.1010101010101	0.781767862399903\\
2.14141414141414	0.76660218834497\\
2.18181818181818	0.746837947507958\\
2.22222222222222	0.722847659765487\\
2.26262626262626	0.695074350302397\\
2.3030303030303	0.664017959950017\\
2.34343434343434	0.63022041913464\\
2.38383838383838	0.594250006109949\\
2.42424242424242	0.556685616223426\\
2.46464646464646	0.518101543059399\\
2.50505050505051	0.47905331740255\\
2.54545454545455	0.440065070739981\\
2.58585858585859	0.401618792392587\\
2.62626262626263	0.364145740040454\\
2.66666666666667	0.328020149351987\\
2.70707070707071	0.293555276383304\\
2.74747474747475	0.261001702453701\\
2.78787878787879	0.230547740368851\\
2.82828282828283	0.202321706924249\\
2.86868686868687	0.176395771937898\\
2.90909090909091	0.152791059570133\\
2.94949494949495	0.131483662992913\\
2.98989898989899	0.112411237017888\\
3.03030303030303	0.0954798526137071\\
3.07070707070707	0.0805708292326469\\
3.11111111111111	0.0675473020705413\\
3.15151515151515	0.0562603282745569\\
3.19191919191919	0.0465543853321692\\
3.23232323232323	0.0382721634279922\\
3.27272727272727	0.0312585989537112\\
3.31313131313131	0.0253641366983196\\
3.35353535353535	0.0204472422439212\\
3.39393939393939	0.0163762130535745\\
3.43434343434343	0.0130303565045358\\
3.47474747474747	0.0103006159847218\\
3.51515151515152	0.00808973277177296\\
3.55555555555556	0.0063120326328361\\
3.5959595959596	0.00489292293663676\\
3.63636363636364	0.00376817962030763\\
3.67676767676768	0.00288309463307418\\
3.71717171717172	0.00219154442862412\\
3.75757575757576	0.00165502950936388\\
3.7979797979798	0.00124172459833281\\
3.83838383838384	0.000925569228919402\\
3.87878787878788	0.000685419745917184\\
3.91919191919192	0.000504276112305317\\
3.95959595959596	0.000368590604636302\\
4	0.000267660451529771\\
};
\addplot [color=mycolor1, draw=none, mark=o, mark options={solid, mycolor1}, forget plot]
  table[row sep=crcr]{%
2.10486214487437	0\\
2.3174485110034	0\\
2.5311903994273	0\\
2.89462424982559	0\\
2.11384782642133	0\\
1.86481982768848	0\\
1.47056427896252	0\\
1.88559118455983	0\\
1.54873259527504	0\\
1.55799539691512	0\\
};
\end{axis}
\end{tikzpicture}%
			\caption{1D Gaussian distribution}
			\label{fig:gaussian:1d}
		\end{subfigure}%
		\hfill
		\begin{subfigure}[t]{0.3\linewidth}
			% This file was created by matlab2tikz.
%
%The latest updates can be retrieved from
%  http://www.mathworks.com/matlabcentral/fileexchange/22022-matlab2tikz-matlab2tikz
%where you can also make suggestions and rate matlab2tikz.
%
\definecolor{mycolor1}{rgb}{0.63500,0.07800,0.18400}%
%
\begin{tikzpicture}[%
trim axis left,
trim axis right
]

\begin{axis}[%
width=0.951\linewidth,
height=3cm,
at={(0\linewidth,0cm)},
scale only axis,
xmin=-0.978366170242519,
xmax=2.82531481135257,
xtick={-2,-1.5,-1,-0.5,0,0.5,1,1.5,2},
xticklabels={{}},
xlabel style={font=\color{white!15!black}},
xlabel={$x_1$},
ymin=0.5,
ymax=3.5,
ytick={-1,-0.5,0,0.5,1,1.5,2,2.5,3},
yticklabels={{}},
ylabel style={font=\color{white!15!black}},
ylabel={$x_2$},
axis background/.style={fill=white},
axis x line*=bottom,
axis y line*=left
]

\addplot[area legend, draw=none, fill=gray, fill opacity=0.5, forget plot]
table[row sep=crcr] {%
x	y\\
0.768979372139707	1.23699069848947\\
0.726730020205697	1.25019378682406\\
0.68475035274946	1.26413684336585\\
0.643081798634509	1.2788061080014\\
0.601765479693048	1.29418710393756\\
0.560842170143694	1.31026465198831\\
0.52035225635224	1.32702288555479\\
0.480335696975194	1.34444526628371\\
0.440831983525401	1.36251460038874\\
0.401880101398695	1.38121305561863\\
0.363518491400019	1.40052217885557\\
0.325785011806987	1.42042291432615\\
0.28871690100834	1.4408956224071\\
0.252350740754148	1.46192009900724\\
0.216722420054043	1.48347559550649\\
0.181867099759096	1.50554083923223\\
0.147819177862305	1.52809405445287\\
0.114612255551927	1.55111298386787\\
0.0822791040511605	1.574574910573\\
0.0508516322769095	1.59845668047921\\
0.0203608553495296	1.62273472516287\\
-0.00916313601534702	1.647385085125\\
-0.0376912052027187	1.67238343343641\\
-0.0651951984525325	1.69770509974538\\
-0.0916479726440491	1.7233250946244\\
-0.117023422082822	1.74921813423166\\
-0.141296504263872	1.77535866526317\\
-0.16444326458561	1.80172089017078\\
-0.186440859990141	1.82827879262128\\
-0.207267581506603	1.8550061631713\\
-0.2269028756753	1.88187662513295\\
-0.245327364831484	1.9088636606044\\
-0.262522866228774	1.9359406366398\\
-0.278472409983329	1.96308083153285\\
-0.293160255821075	1.99025746118788\\
-0.306571908611456	2.01744370555249\\
-0.318694132672375	2.04461273508571\\
-0.329514964832219	2.07173773723557\\
-0.33902372623606	2.09879194289978\\
-0.3472110328844	2.12574865284365\\
-0.354068804894042	2.15258126404898\\
-0.359590274471962	2.17926329596804\\
-0.363769992594301	2.20576841665671\\
-0.366603834383896	2.23207046876086\\
-0.368089003181034	2.25814349533055\\
-0.36822403330342	2.2839617654364\\
-0.367008791492625	2.30949979956287\\
-0.364444477045598	2.33473239475354\\
-0.360533620631105	2.35963464948332\\
-0.355280081792263	2.38418198823328\\
-0.34868904513764	2.4083501857437\\
-0.340767015224668	2.43211539092152\\
-0.331521810140432	2.45545415037841\\
-0.320962553786165	2.47834343157653\\
-0.309099666873052	2.50076064555878\\
-0.295944856638253	2.52268366924143\\
-0.281511105291272	2.54409086724691\\
-0.265812657202078	2.5649611132553\\
-0.248865004843641	2.58527381085349\\
-0.230684873502721	2.60500891386133\\
-0.21129020477404	2.6241469461148\\
-0.190700138854086	2.64266902068663\\
-0.168934995652053	2.66055685852541\\
-0.146016254736543	2.67779280649482\\
-0.121966534137828	2.69435985479514\\
-0.0968095680265795	2.71024165374986\\
-0.0705701832911125	2.72542252994084\\
-0.0432742750362425	2.73988750167609\\
-0.0149487810279409	2.75362229377489\\
0.0143783448909837	2.76661335165566\\
0.0446781603879562	2.77884785471269\\
0.0759207632032231	2.79031372896856\\
0.108075320659769	2.80099965898968\\
0.141110100091454	2.81089509905323\\
0.174992500159313	2.81999028355457\\
0.209689083025152	2.82827623664466\\
0.245165607350653	2.83574478108819\\
0.281387062089437	2.84238854633348\\
0.318317701038748	2.84820097578634\\
0.355921078116633	2.85317633328064\\
0.394160083329825	2.85730970873921\\
0.432996979396832	2.86059702301952\\
0.472393438990069	2.86303503193925\\
0.512310582560312	2.86462132947794\\
0.55270901670611	2.86535435015142\\
0.593548873050325	2.86523337055681\\
0.6347898475854	2.86425851008635\\
0.676391240448551	2.86243073080962\\
0.718311996087622	2.85975183652411\\
0.760510743777949	2.85622447097505\\
0.802945838450278	2.85185211524637\\
0.84557540178942	2.84663908432529\\
0.888357363563088	2.84059052284396\\
0.931249503140141	2.83371240000231\\
0.974209491157246	2.82601150367717\\
1.01719493129285	2.81749543372348\\
1.06016340210724	2.80817259447413\\
1.10307249890737	2.79805218644588\\
1.14587987559517	2.78714419725963\\
1.18854328645804	2.77545939178375\\
1.23102062786029	2.76300930151053\\
1.2732699797943	2.74980621317594\\
1.31524964725054	2.73586315663415\\
1.35691820136549	2.7211938919986\\
1.39823452030695	2.70581289606244\\
1.43915782985631	2.68973534801169\\
1.47964774364776	2.67297711444521\\
1.51966430302481	2.65555473371629\\
1.5591680164746	2.63748539961126\\
1.5981198986013	2.61878694438137\\
1.63648150859998	2.59947782114443\\
1.67421498819301	2.57957708567385\\
1.71128309899166	2.5591043775929\\
1.74764925924585	2.53807990099276\\
1.78327757994596	2.51652440449351\\
1.8181329002409	2.49445916076777\\
1.85218082213769	2.47190594554713\\
1.88538774444807	2.44888701613213\\
1.91772089594884	2.425425089427\\
1.94914836772309	2.40154331952079\\
1.97963914465047	2.37726527483713\\
2.00916313601535	2.352614914875\\
2.03769120520272	2.3276165665636\\
2.06519519845253	2.30229490025462\\
2.09164797264405	2.2766749053756\\
2.11702342208282	2.25078186576834\\
2.14129650426387	2.22464133473683\\
2.16444326458561	2.19827910982922\\
2.18644085999014	2.17172120737872\\
2.2072675815066	2.1449938368287\\
2.2269028756753	2.11812337486705\\
2.24532736483148	2.0911363393956\\
2.26252286622877	2.0640593633602\\
2.27847240998333	2.03691916846715\\
2.29316025582107	2.00974253881212\\
2.30657190861146	1.98255629444751\\
2.31869413267238	1.95538726491429\\
2.32951496483222	1.92826226276443\\
2.33902372623606	1.90120805710022\\
2.3472110328844	1.87425134715635\\
2.35406880489404	1.84741873595103\\
2.35959027447196	1.82073670403196\\
2.3637699925943	1.79423158334329\\
2.3666038343839	1.76792953123914\\
2.36808900318103	1.74185650466945\\
2.36822403330342	1.7160382345636\\
2.36700879149262	1.69050020043713\\
2.3644444770456	1.66526760524646\\
2.3605336206311	1.64036535051668\\
2.35528008179226	1.61581801176672\\
2.34868904513764	1.5916498142563\\
2.34076701522467	1.56788460907849\\
2.33152181014043	1.54454584962159\\
2.32096255378617	1.52165656842347\\
2.30909966687305	1.49923935444122\\
2.29594485663825	1.47731633075857\\
2.28151110529127	1.45590913275309\\
2.26581265720208	1.4350388867447\\
2.24886500484364	1.41472618914651\\
2.23068487350272	1.39499108613867\\
2.21129020477404	1.3758530538852\\
2.19070013885409	1.35733097931338\\
2.16893499565205	1.33944314147459\\
2.14601625473654	1.32220719350518\\
2.12196653413783	1.30564014520486\\
2.09680956802658	1.28975834625014\\
2.07057018329111	1.27457747005916\\
2.04327427503624	1.26011249832391\\
2.01494878102794	1.24637770622511\\
1.98562165510902	1.23338664834434\\
1.95532183961204	1.22115214528731\\
1.92407923679678	1.20968627103144\\
1.89192467934023	1.19900034101032\\
1.85888989990855	1.18910490094677\\
1.82500749984069	1.18000971644543\\
1.79031091697485	1.17172376335534\\
1.75483439264935	1.16425521891181\\
1.71861293791056	1.15761145366652\\
1.68168229896125	1.15179902421366\\
1.64407892188337	1.14682366671936\\
1.60583991667018	1.14269029126079\\
1.56700302060317	1.13940297698048\\
1.52760656100993	1.13696496806075\\
1.48768941743969	1.13537867052207\\
1.44729098329389	1.13464564984858\\
1.40645112694968	1.13476662944319\\
1.3652101524146	1.13574148991365\\
1.32360875955145	1.13756926919038\\
1.28168800391238	1.14024816347589\\
1.23948925622205	1.14377552902495\\
1.19705416154972	1.14814788475363\\
1.15442459821058	1.15336091567471\\
1.11164263643691	1.15940947715604\\
1.06875049685986	1.16628759999769\\
1.02579050884275	1.17398849632283\\
0.982805068707147	1.18250456627652\\
0.939836597892757	1.19182740552587\\
0.896927501092631	1.20194781355412\\
0.854120124404832	1.21285580274037\\
0.811456713541956	1.22454060821625\\
0.768979372139707	1.23699069848947\\
}--cycle;
\addplot [color=mycolor1, draw=none, mark=o, mark options={solid, mycolor1}, forget plot]
  table[row sep=crcr]{%
1.62168985772206	1.23319741840057\\
-0.587602734884004	2.69550462158856\\
0.654412359379664	2.24916339882187\\
0.212027330163576	2.40979113724902\\
2.23708595224499	1.39906225859969\\
0.949399287486024	2.13161156498154\\
0.47402533634125	2.48475995719116\\
0.804673942880572	2.12421668460537\\
0.419678263290235	2.4155337431119\\
0.870160557857379	2.05692839835231\\
0.564588505601391	2.22420707552924\\
1.29124732916342	2.18692999207973\\
1.31507621361342	1.60279946697807\\
1.3035525493338	2.31257712708344\\
1.1495798927265	1.80744863845831\\
0.371725611929442	2.33558117665467\\
1.12732693572524	1.67585886862615\\
1.22697793180628	1.69968063090332\\
0.0791835638554578	1.98103195915242\\
1.44121236292173	2.10265261413699\\
1.18517475716689	1.4187349367444\\
1.50343148953022	1.77117958688951\\
1.34789370954102	1.73144646175673\\
0.786613620468095	2.27584290790401\\
2.43455137599406	2.06388386700375\\
1.2134890945905	2.03443026268702\\
1.0611784032452	2.0205199638735\\
0.731083846566839	1.40104149090502\\
1.94552345481951	2.13883935233326\\
0.479603449359114	1.88512026660411\\
1.34791998024851	1.86675300099744\\
1.13337542611373	1.81614086588043\\
1.74514836450965	2.67761202921273\\
2.39595683555146	1.53478272330273\\
1.63727722223667	1.788012157073\\
1.17040538445556	2.25348062671072\\
1.03627772619526	1.72185695316155\\
1.36779663288073	2.10449285814305\\
1.01674936612708	1.84654258927429\\
1.43491401911157	2.29499423790677\\
1.60619456014794	2.13429642639953\\
1.11386864264935	3.11900306041096\\
1.25526263382011	2.07473885125444\\
1.02230851133055	2.0176376482359\\
1.55887259687329	1.42871807096233\\
-0.238078139405796	2.42282794792377\\
0.800094693378852	2.15817667293352\\
0.726871306657918	2.41303519150767\\
0.555644195614291	2.14916985224035\\
0.842651423931794	1.88818962077398\\
};
\end{axis}
\end{tikzpicture}%
			\caption{2D Gaussian distribution}
			\label{fig:gaussian:2d}
		\end{subfigure}%
		\hfill
		\begin{subfigure}[t]{0.3\linewidth}
			% This file was created by matlab2tikz.
%
%The latest updates can be retrieved from
%  http://www.mathworks.com/matlabcentral/fileexchange/22022-matlab2tikz-matlab2tikz
%where you can also make suggestions and rate matlab2tikz.
%
\definecolor{mycolor1}{rgb}{0.63500,0.07800,0.18400}%
%
\begin{tikzpicture}[%
trim axis left,
trim axis right
]

\begin{axis}[%
width=0.951\linewidth,
height=3cm,
at={(0\linewidth,0cm)},
scale only axis,
xmin=0,
xmax=4,
xtick={0,0.5,1,1.5,2,2.5,3,3.5,4},
xticklabels={{}},
xlabel style={font=\color{white!15!black}},
xlabel={$i$},
ymin=-2,
ymax=2,
ytick={-2,-1.5,-1,-0.5,0,0.5,1,1.5,2},
yticklabels={{}},
ylabel style={font=\color{white!15!black}},
ylabel={$x(i)$},
axis background/.style={fill=white},
axis x line*=bottom,
axis y line*=left
]

\addplot[area legend, draw=none, fill=gray, fill opacity=0.5, forget plot]
table[row sep=crcr] {%
x	y\\
0	1.33894093138436\\
0.0404040404040404	1.30759268640533\\
0.0808080808080808	1.27416049771751\\
0.121212121212121	1.23860882221759\\
0.161616161616162	1.20091109390162\\
0.202020202020202	1.1610503566414\\
0.242424242424242	1.1190198386713\\
0.282828282828283	1.07482346168818\\
0.323232323232323	1.02847627816469\\
0.363636363636364	0.980004831260679\\
0.404040404040404	0.929447432576734\\
0.444444444444444	0.876854353916285\\
0.484848484848485	0.822287930194645\\
0.525252525252525	0.765822571640604\\
0.565656565656566	0.707544684463923\\
0.606060606060606	0.647552500194751\\
0.646464646464647	0.585955814922915\\
0.686868686868687	0.522875640660376\\
0.727272727272727	0.458443772003447\\
0.767676767676768	0.392802272167531\\
0.808080808080808	0.326102883292137\\
0.848484848484849	0.258506366654687\\
0.888888888888889	0.190181779076716\\
0.929292929292929	0.121305692345679\\
0.96969696969697	0.0520613629021163\\
1.01010101010101	0.0145991338972249\\
1.05050505050505	0.0723766497107184\\
1.09090909090909	0.129003620555568\\
1.13131313131313	0.18426700099537\\
1.17171717171717	0.237955319599258\\
1.21212121212121	0.289859890086144\\
1.25252525252525	0.339776008540474\\
1.29292929292929	0.38750412781218\\
1.33333333333333	0.432851000681895\\
1.37373737373737	0.475630783918306\\
1.41414141414141	0.515666095962986\\
1.45454545454545	0.55278902163469\\
1.4949494949495	0.586842057934705\\
1.53535353535354	0.617678995741859\\
1.57575757575758	0.645165732895641\\
1.61616161616162	0.669181014864292\\
1.65656565656566	0.68961709986907\\
1.6969696969697	0.706380345974817\\
1.73737373737374	0.719391718251518\\
1.77777777777778	0.728587214654395\\
1.81818181818182	0.733918209756796\\
1.85858585858586	0.735351715898536\\
1.8989898989899	0.732870561682673\\
1.93939393939394	0.72647348806883\\
1.97979797979798	0.716175162576002\\
2.02020202020202	0.702006112329395\\
2.06060606060606	0.684012576872861\\
2.1010101010101	0.662256281830744\\
2.14141414141414	0.63681413465141\\
2.18181818181818	0.607777843810091\\
2.22222222222222	0.575253463002362\\
2.26262626262626	0.539360862031278\\
2.3030303030303	0.500233126290724\\
2.34343434343434	0.458015886982118\\
2.38383838383838	0.412866584477157\\
2.42424242424242	0.364953667559814\\
2.46464646464646	0.314455731647025\\
2.50505050505051	0.26156059949887\\
2.54545454545455	0.206464348381622\\
2.58585858585859	0.14937028813506\\
2.62626262626263	0.0904878951105028\\
2.66666666666667	0.0300317074782748\\
2.70707070707071	-0.0317798120592501\\
2.74747474747475	-0.0947254395834641\\
2.78787878787879	-0.158582372230257\\
2.82828282828283	-0.223127421870649\\
2.86868686868687	-0.288138212233831\\
2.90909090909091	-0.353394371148582\\
2.94949494949495	-0.418678709298082\\
2.98989898989899	-0.483778376691341\\
3.03030303030303	-0.452309101810315\\
3.07070707070707	-0.387740244453384\\
3.11111111111111	-0.322239400460619\\
3.15151515151515	-0.256012423402569\\
3.19191919191919	-0.1892636024007\\
3.23232323232323	-0.122194577988257\\
3.27272727272727	-0.0550032802017414\\
3.31313131313131	0.0121171015161398\\
3.35353535353535	0.0789791072211162\\
3.39393939393939	0.145401945135791\\
3.43434343434343	0.211212371510147\\
3.47474747474747	0.276245503699543\\
3.51515151515152	0.34034555920001\\
3.55555555555556	0.403366514202103\\
3.5959595959596	0.465172676138444\\
3.63636363636364	0.52563916569356\\
3.67676767676768	0.584652304801634\\
3.71717171717172	0.642109908262576\\
3.75757575757576	0.697921477741728\\
3.7979797979798	0.752008298067845\\
3.83838383838384	0.804303436889115\\
3.87878787878788	0.854751649872262\\
3.91919191919192	0.903309194717945\\
3.95959595959596	0.94994355830152\\
4	0.99463310221609\\
4	-1.62056574492566\\
3.95959595959596	-1.60014864510581\\
3.91919191919192	-1.57768008100027\\
3.87878787878788	-1.55309506793653\\
3.83838383838384	-1.52633699011473\\
3.7979797979798	-1.49735844251796\\
3.75757575757576	-1.46612202552858\\
3.71717171717172	-1.4326010830895\\
3.67676767676768	-1.39678037574202\\
3.63636363636364	-1.35865668047412\\
3.5959595959596	-1.31823931001742\\
3.55555555555556	-1.27555054502897\\
3.51515151515152	-1.23062597347478\\
3.47474747474747	-1.1835147324855\\
3.43434343434343	-1.13427964896632\\
3.39393939393939	-1.08299727630139\\
3.35353535353535	-1.02975782558188\\
3.31313131313131	-0.974664990891168\\
3.27272727272727	-0.917835669285202\\
3.23232323232323	-0.85939957719499\\
3.19191919191919	-0.799498766034983\\
3.15151515151515	-0.738287040810975\\
3.11111111111111	-0.675929286470618\\
3.07070707070707	-0.612600707613061\\
3.03030303030303	-0.548485987961175\\
2.98989898989899	-0.51573965123878\\
2.94949494949495	-0.577826203477327\\
2.90909090909091	-0.638364794722935\\
2.86868686868687	-0.697154264729565\\
2.82828282828283	-0.753997311032638\\
2.78787878787879	-0.808701511331117\\
2.74747474747475	-0.861080306726273\\
2.70707070707071	-0.910953939265186\\
2.66666666666667	-0.958150338048989\\
2.62626262626263	-1.00250594902311\\
2.58585858585859	-1.0438665044481\\
2.54545454545455	-1.08208772893851\\
2.50505050505051	-1.11703597983099\\
2.46464646464646	-1.1485888204854\\
2.42424242424242	-1.17663552591466\\
2.38383838383838	-1.20107752086529\\
2.34343434343434	-1.22182875111745\\
2.3030303030303	-1.23881598932836\\
2.26262626262626	-1.25197907719923\\
2.22222222222222	-1.26127110609627\\
2.18181818181818	-1.26665853850048\\
2.14141414141414	-1.2681212727979\\
2.1010101010101	-1.26565265395838\\
2.06060606060606	-1.25925943259261\\
2.02020202020202	-1.24896167473509\\
1.97979797979798	-1.23479262448848\\
1.93939393939394	-1.21679852139665\\
1.8989898989899	-1.19503837410645\\
1.85858585858586	-1.16958369155077\\
1.81818181818182	-1.14051817255378\\
1.77777777777778	-1.10793735444424\\
1.73737373737374	-1.07194822097898\\
1.6969696969697	-1.03266876964427\\
1.65656565656566	-0.990227538230495\\
1.61616161616162	-0.944763090478155\\
1.57575757575758	-0.896423460578834\\
1.53535353535354	-0.845365556390569\\
1.4949494949495	-0.791754521395153\\
1.45454545454545	-0.735763055685444\\
1.41414141414141	-0.677570696620178\\
1.37373737373737	-0.617363060215303\\
1.33333333333333	-0.555331044845371\\
1.29292929292929	-0.491669999393737\\
1.25252525252525	-0.426578858602334\\
1.21212121212121	-0.360259249014704\\
1.17171717171717	-0.292914569562722\\
1.13131313131313	-0.224749051500359\\
1.09090909090909	-0.155966803018742\\
1.05050505050505	-0.0867708444685266\\
1.01010101010101	-0.0173621406503643\\
0.96969696969697	-0.0441155232487941\\
0.929292929292929	-0.103554770813987\\
0.888888888888889	-0.163508106933296\\
0.848484848484849	-0.223768250753692\\
0.808080808080808	-0.28413228034215\\
0.767676767676768	-0.344402727039222\\
0.727272727272727	-0.404388617079996\\
0.686868686868687	-0.463906451746933\\
0.646464646464647	-0.522781117880096\\
0.606060606060606	-0.580846721242423\\
0.565656565656566	-0.637947336012539\\
0.525252525252525	-0.693937664544437\\
0.484848484848485	-0.748683602480138\\
0.444444444444444	-0.802062705314783\\
0.404040404040404	-0.853964553579141\\
0.363636363636364	-0.904291014907002\\
0.323232323232323	-0.952956402378972\\
0.282828282828283	-0.999887529663901\\
0.242424242424242	-1.04502366459901\\
0.202020202020202	-1.08831638394441\\
0.161616161616162	-1.12972933310222\\
0.121212121212121	-1.1692378955912\\
0.0808080808080808	-1.20682877800071\\
0.0404040404040404	-1.24249951700201\\
0	-1.27625791575739\\
}--cycle;
\addplot [color=mycolor1, forget plot]
  table[row sep=crcr]{%
0	0.478469582417546\\
0.0404040404040404	0.457518910022232\\
0.0808080808080808	0.433492092958278\\
0.121212121212121	0.407088662415206\\
0.161616161616162	0.379731942281659\\
0.202020202020202	0.349658076412809\\
0.242424242424242	0.318887634555395\\
0.282828282828283	0.288070363788865\\
0.323232323232323	0.255910086724378\\
0.363636363636364	0.224134578367703\\
0.404040404040404	0.193852043804622\\
0.444444444444444	0.163914909263444\\
0.484848484848485	0.134931835427498\\
0.525252525252525	0.107291352594057\\
0.565656565656566	0.0822667579459298\\
0.606060606060606	0.0602578875944031\\
0.646464646464647	0.0402301694923436\\
0.686868686868687	0.0229658269354559\\
0.727272727272727	0.00842623195449402\\
0.767676767676768	-0.00285947541326662\\
0.808080808080808	-0.0102504319901325\\
0.848484848484849	-0.0144095723270378\\
0.888888888888889	-0.0158047765496966\\
0.929292929292929	-0.0123358334269094\\
0.96969696969697	-0.00634395976973593\\
1.01010101010101	0.00322018711850536\\
1.05050505050505	0.0152144225269035\\
1.09090909090909	0.0304199513607418\\
1.13131313131313	0.0480823714583319\\
1.17171717171717	0.0686693226052524\\
1.21212121212121	0.0921542819127697\\
1.25252525252525	0.116798647262324\\
1.29292929292929	0.143813509188276\\
1.33333333333333	0.171857843378672\\
1.37373737373737	0.201280335450405\\
1.41414141414141	0.230998406527799\\
1.45454545454545	0.261173673374284\\
1.4949494949495	0.291439022678276\\
1.53535353535354	0.322230743570669\\
1.57575757575758	0.351545286968389\\
1.61616161616162	0.37964692360427\\
1.65656565656566	0.406855452959618\\
1.6969696969697	0.432188700595606\\
1.73737373737374	0.455575945253821\\
1.77777777777778	0.476804042959088\\
1.81818181818182	0.495229853362924\\
1.85858585858586	0.511048227421426\\
1.8989898989899	0.522791438365154\\
1.93939393939394	0.531306647255465\\
1.97979797979798	0.536846921019509\\
2.02020202020202	0.537736044116635\\
2.06060606060606	0.534913075105064\\
2.1010101010101	0.527757845164547\\
2.14141414141414	0.516487946938664\\
2.18181818181818	0.500372710062048\\
2.22222222222222	0.480921940283368\\
2.26262626262626	0.455810490697506\\
2.3030303030303	0.427097830404977\\
2.34343434343434	0.393215911114766\\
2.38383838383838	0.356002304585787\\
2.42424242424242	0.314416337825185\\
2.46464646464646	0.269342398259971\\
2.50505050505051	0.220683524810998\\
2.54545454545455	0.168814925707719\\
2.58585858585859	0.11466808036793\\
2.62626262626263	0.0581630886210114\\
2.66666666666667	-0.00185236644286002\\
2.70707070707071	-0.0618686982965438\\
2.74747474747475	-0.123164277871589\\
2.78787878787879	-0.186148778275804\\
2.82828282828283	-0.248284445107553\\
2.86868686868687	-0.309623729627961\\
2.90909090909091	-0.369904581397221\\
2.94949494949495	-0.429372635808935\\
2.98989898989899	-0.486438187773288\\
3.03030303030303	-0.540260392350545\\
3.07070707070707	-0.591511453324493\\
3.11111111111111	-0.639588149711263\\
3.15151515151515	-0.682568865995513\\
3.19191919191919	-0.722291247302741\\
3.23232323232323	-0.756872707532002\\
3.27272727272727	-0.785554842272102\\
3.31313131313131	-0.809318450689232\\
3.35353535353535	-0.827985167063362\\
3.39393939393939	-0.840549698624765\\
3.43434343434343	-0.848018699935092\\
3.47474747474747	-0.84835244467295\\
3.51515151515152	-0.842996604692319\\
3.55555555555556	-0.831497075484333\\
3.5959595959596	-0.814430546181353\\
3.63636363636364	-0.791438256146627\\
3.67676767676768	-0.763256390126166\\
3.71717171717172	-0.730354496508202\\
3.75757575757576	-0.692083074205149\\
3.7979797979798	-0.648936022001407\\
3.83838383838384	-0.60168193635539\\
3.87878787878788	-0.55040277677957\\
3.91919191919192	-0.495945067983782\\
3.95959595959596	-0.438162948673068\\
4	-0.377731363252925\\
};
\addplot [color=mycolor1, forget plot]
  table[row sep=crcr]{%
0	1.60284826614559\\
0.0404040404040404	1.54092822349489\\
0.0808080808080808	1.4731358225784\\
0.121212121212121	1.40044839613468\\
0.161616161616162	1.32471602432669\\
0.202020202020202	1.24622141462148\\
0.242424242424242	1.16435667143981\\
0.282828282828283	1.08184481350115\\
0.323232323232323	0.998448816418469\\
0.363636363636364	0.915146757791022\\
0.404040404040404	0.831705424826958\\
0.444444444444444	0.750355819930581\\
0.484848484848485	0.670776014531228\\
0.525252525252525	0.593878397907915\\
0.565656565656566	0.520626642794438\\
0.606060606060606	0.450259394618403\\
0.646464646464647	0.384006077491542\\
0.686868686868687	0.322450565664098\\
0.727272727272727	0.26483043773382\\
0.767676767676768	0.211160766086137\\
0.808080808080808	0.16358165134981\\
0.848484848484849	0.120481412709561\\
0.888888888888889	0.0821656738820477\\
0.929292929292929	0.0484383412962266\\
0.96969696969697	0.0188799256673567\\
1.01010101010101	-0.0060933697229457\\
1.05050505050505	-0.0264446150206678\\
1.09090909090909	-0.0443120978217886\\
1.13131313131313	-0.058746070668303\\
1.17171717171717	-0.0695444262466091\\
1.21212121212121	-0.0782152791103577\\
1.25252525252525	-0.0850979294556543\\
1.29292929292929	-0.090178985285012\\
1.33333333333333	-0.0940666424420017\\
1.37373737373737	-0.0968727246499494\\
1.41414141414141	-0.0990938355645731\\
1.45454545454545	-0.102456045996181\\
1.4949494949495	-0.106074406078573\\
1.53535353535354	-0.109520613414642\\
1.57575757575758	-0.114709408433973\\
1.61616161616162	-0.120979326212659\\
1.65656565656566	-0.128729805982721\\
1.6969696969697	-0.137693420016658\\
1.73737373737374	-0.150143112853455\\
1.77777777777778	-0.163001248151151\\
1.81818181818182	-0.17754413131811\\
1.85858585858586	-0.194379977244496\\
1.8989898989899	-0.212059037985923\\
1.93939393939394	-0.231600030646979\\
1.97979797979798	-0.253691564613625\\
2.02020202020202	-0.276106373183013\\
2.06060606060606	-0.299419495988814\\
2.1010101010101	-0.323604814967954\\
2.14141414141414	-0.348464954576695\\
2.18181818181818	-0.374084232618906\\
2.22222222222222	-0.399016122031278\\
2.26262626262626	-0.422872637997563\\
2.3030303030303	-0.446540677681178\\
2.34343434343434	-0.469700483957387\\
2.38383838383838	-0.491115513900016\\
2.42424242424242	-0.510291118400803\\
2.46464646464646	-0.527778090031495\\
2.50505050505051	-0.542858715158121\\
2.54545454545455	-0.556226982619163\\
2.58585858585859	-0.567100654918981\\
2.62626262626263	-0.573872157888552\\
2.66666666666667	-0.578975382414508\\
2.70707070707071	-0.58085704640769\\
2.74747474747475	-0.579292669234211\\
2.78787878787879	-0.57479026747343\\
2.82828282828283	-0.567140962740885\\
2.86868686868687	-0.555374330416708\\
2.90909090909091	-0.542438494642446\\
2.94949494949495	-0.525508387711043\\
2.98989898989899	-0.505485893229694\\
3.03030303030303	-0.483643188848412\\
3.07070707070707	-0.458196832632069\\
3.11111111111111	-0.431236352608885\\
3.15151515151515	-0.402696845437628\\
3.19191919191919	-0.371513752883716\\
3.23232323232323	-0.3394257171485\\
3.27272727272727	-0.305886767818594\\
3.31313131313131	-0.271391165008437\\
3.35353535353535	-0.237380157947724\\
3.39393939393939	-0.20168407717468\\
3.43434343434343	-0.167365585458744\\
3.47474747474747	-0.132377976657531\\
3.51515151515152	-0.0966671877874758\\
3.55555555555556	-0.0628233900622073\\
3.5959595959596	-0.0300673478639745\\
3.63636363636364	0.00175502354612372\\
3.67676767676768	0.0325859812240233\\
3.71717171717172	0.0626187180902162\\
3.75757575757576	0.0905617801631873\\
3.7979797979798	0.118658319668861\\
3.83838383838384	0.142776588052411\\
3.87878787878788	0.166975947591472\\
3.91919191919192	0.18887040207151\\
3.95959595959596	0.208933353040482\\
4	0.227624712408493\\
};
\addplot [color=mycolor1, forget plot]
  table[row sep=crcr]{%
0	-0.456929606597449\\
0.0404040404040404	-0.446312293143547\\
0.0808080808080808	-0.432929132253893\\
0.121212121212121	-0.416589311465665\\
0.161616161616162	-0.39844605544801\\
0.202020202020202	-0.377163600580906\\
0.242424242424242	-0.354209106815802\\
0.282828282828283	-0.329668165984978\\
0.323232323232323	-0.303401766566019\\
0.363636363636364	-0.276626166044152\\
0.404040404040404	-0.248734045967304\\
0.444444444444444	-0.22092305203162\\
0.484848484848485	-0.192708344004052\\
0.525252525252525	-0.165983191393554\\
0.565656565656566	-0.138977307067807\\
0.606060606060606	-0.114165615676062\\
0.646464646464647	-0.0900998511262302\\
0.686868686868687	-0.068491911670428\\
0.727272727272727	-0.0488256794864739\\
0.767676767676768	-0.0322530776033763\\
0.808080808080808	-0.017933133332047\\
0.848484848484849	-0.00822632399798229\\
0.888888888888889	-0.00095932886304898\\
0.929292929292929	0.00297790343148032\\
0.96969696969697	0.00249447160398401\\
1.01010101010101	-0.00152302349082493\\
1.05050505050505	-0.0094508832539407\\
1.09090909090909	-0.0217828776881429\\
1.13131313131313	-0.0384581406063199\\
1.17171717171717	-0.0589003279198144\\
1.21212121212121	-0.0831421539354623\\
1.25252525252525	-0.111395381371072\\
1.29292929292929	-0.142912734008764\\
1.33333333333333	-0.179013716923604\\
1.37373737373737	-0.218286700789947\\
1.41414141414141	-0.260870867600472\\
1.45454545454545	-0.305014035842998\\
1.4949494949495	-0.353360344404165\\
1.53535353535354	-0.403167251731865\\
1.57575757575758	-0.454603350394991\\
1.61616161616162	-0.507900812484219\\
1.65656565656566	-0.561131334972557\\
1.6969696969697	-0.616329013173054\\
1.73737373737374	-0.670645624502317\\
1.77777777777778	-0.725210898335254\\
1.81818181818182	-0.777811156556149\\
1.85858585858586	-0.830521478555083\\
1.8989898989899	-0.880461572118521\\
1.93939393939394	-0.928107420913653\\
1.97979797979798	-0.972646927793162\\
2.02020202020202	-1.01459251956616\\
2.06060606060606	-1.05314712115928\\
2.1010101010101	-1.08727876703118\\
2.14141414141414	-1.11693788612778\\
2.18181818181818	-1.14164512269879\\
2.22222222222222	-1.16146209966038\\
2.26262626262626	-1.17600763243626\\
2.3030303030303	-1.18522670596073\\
2.34343434343434	-1.18833813353884\\
2.38383838383838	-1.1854994537439\\
2.42424242424242	-1.17701698598808\\
2.46464646464646	-1.16195970751012\\
2.50505050505051	-1.1415976104845\\
2.54545454545455	-1.11471100416958\\
2.58585858585859	-1.08314638934719\\
2.62626262626263	-1.04487827693125\\
2.66666666666667	-1.00174020528718\\
2.70707070707071	-0.953830345140852\\
2.74747474747475	-0.901360599822058\\
2.78787878787879	-0.845479952731861\\
2.82828282828283	-0.784728601034326\\
2.86868686868687	-0.721449238440853\\
2.90909090909091	-0.655052838397982\\
2.94949494949495	-0.587338846111631\\
2.98989898989899	-0.517182218925174\\
3.03030303030303	-0.447481902120993\\
3.07070707070707	-0.376286020111626\\
3.11111111111111	-0.305983137407256\\
3.15151515151515	-0.236558075602783\\
3.19191919191919	-0.16869519660492\\
3.23232323232323	-0.102446790650735\\
3.27272727272727	-0.0393207441556837\\
3.31313131313131	0.0210014885340056\\
3.35353535353535	0.0764500261436331\\
3.39393939393939	0.128969982125645\\
3.43434343434343	0.177499176749244\\
3.47474747474747	0.219945222366529\\
3.51515151515152	0.258732125088174\\
3.55555555555556	0.29145031943283\\
3.5959595959596	0.318369988727293\\
3.63636363636364	0.340097453498102\\
3.67676767676768	0.356594868593564\\
3.71717171717172	0.367258689707463\\
3.75757575757576	0.373112268122919\\
3.7979797979798	0.372601930019776\\
3.83838383838384	0.367871534793177\\
3.87878787878788	0.358036015640045\\
3.91919191919192	0.343732393854908\\
3.95959595959596	0.326117049168531\\
4	0.30363685327344\\
};
\addplot [color=mycolor1, forget plot]
  table[row sep=crcr]{%
0	-0.383560292859447\\
0.0404040404040404	-0.361022306871714\\
0.0808080808080808	-0.339073495517844\\
0.121212121212121	-0.316678138261749\\
0.161616161616162	-0.293788293757621\\
0.202020202020202	-0.272657650613061\\
0.242424242424242	-0.250339240561271\\
0.282828282828283	-0.228190766704437\\
0.323232323232323	-0.207113734585483\\
0.363636363636364	-0.185177757732198\\
0.404040404040404	-0.165125851679926\\
0.444444444444444	-0.146094474798263\\
0.484848484848485	-0.126645302494828\\
0.525252525252525	-0.108519834460572\\
0.565656565656566	-0.0924075673165329\\
0.606060606060606	-0.0753021386835677\\
0.646464646464647	-0.0609360071170793\\
0.686868686868687	-0.0476821715751383\\
0.727272727272727	-0.0359648434953495\\
0.767676767676768	-0.0257093361377663\\
0.808080808080808	-0.0167638833220546\\
0.848484848484849	-0.0096805508886784\\
0.888888888888889	-0.00370704634284668\\
0.929292929292929	-0.00105957918474385\\
0.96969696969697	0.000353287046254102\\
1.01010101010101	-0.000724670888958412\\
1.05050505050505	-0.00230011107226139\\
1.09090909090909	-0.00810317108463155\\
1.13131313131313	-0.0149009198943713\\
1.17171717171717	-0.024077937533982\\
1.21212121212121	-0.0357409161318387\\
1.25252525252525	-0.0488817819333568\\
1.29292929292929	-0.0644777379330287\\
1.33333333333333	-0.0811846009283008\\
1.37373737373737	-0.101113951453871\\
1.41414141414141	-0.122622608289235\\
1.45454545454545	-0.144425831354369\\
1.4949494949495	-0.168596061837203\\
1.53535353535354	-0.193933744212677\\
1.57575757575758	-0.221101497911791\\
1.61616161616162	-0.248565447582038\\
1.65656565656566	-0.276973905651259\\
1.6969696969697	-0.305733917300268\\
1.73737373737374	-0.334852540756359\\
1.77777777777778	-0.364392427003864\\
1.81818181818182	-0.39357951586657\\
1.85858585858586	-0.42184643867831\\
1.8989898989899	-0.450671451156834\\
1.93939393939394	-0.478115174823711\\
1.97979797979798	-0.503946104814833\\
2.02020202020202	-0.529264902386177\\
2.06060606060606	-0.553009632878169\\
2.1010101010101	-0.576147133224652\\
2.14141414141414	-0.596820561913362\\
2.18181818181818	-0.615900056401274\\
2.22222222222222	-0.633141785499506\\
2.26262626262626	-0.647756223688879\\
2.3030303030303	-0.660656899394024\\
2.34343434343434	-0.671739250133647\\
2.38383838383838	-0.679728769690146\\
2.42424242424242	-0.685595926040022\\
2.46464646464646	-0.690120171625768\\
2.50505050505051	-0.692145439342813\\
2.54545454545455	-0.691071352359537\\
2.58585858585859	-0.687630673661646\\
2.62626262626263	-0.681424335629138\\
2.66666666666667	-0.672274817908068\\
2.70707070707071	-0.661743345008715\\
2.74747474747475	-0.647984608351802\\
2.78787878787879	-0.631321734831244\\
2.82828282828283	-0.611855062252542\\
2.86868686868687	-0.59009114033956\\
2.90909090909091	-0.566023240494587\\
2.94949494949495	-0.53782084152921\\
2.98989898989899	-0.507684495097145\\
3.03030303030303	-0.474710666617487\\
3.07070707070707	-0.43934832925032\\
3.11111111111111	-0.400308526969703\\
3.15151515151515	-0.359597555849291\\
3.19191919191919	-0.315794556419639\\
3.23232323232323	-0.26966166719996\\
3.27272727272727	-0.221555292748092\\
3.31313131313131	-0.171024823495423\\
3.35353535353535	-0.119141611516546\\
3.39393939393939	-0.0654680411601681\\
3.43434343434343	-0.0093733071552215\\
3.47474747474747	0.0463619310712045\\
3.51515151515152	0.102750729318479\\
3.55555555555556	0.160255960245136\\
3.5959595959596	0.216854294543502\\
3.63636363636364	0.272442060542777\\
3.67676767676768	0.327356421321481\\
3.71717171717172	0.380923420700821\\
3.75757575757576	0.431762262251814\\
3.7979797979798	0.480888160516654\\
3.83838383838384	0.527117499131354\\
3.87878787878788	0.568709246082861\\
3.91919191919192	0.606815108264737\\
3.95959595959596	0.640077588070883\\
4	0.668977403563912\\
};
\addplot [color=mycolor1, forget plot]
  table[row sep=crcr]{%
0	-0.566824246814511\\
0.0404040404040404	-0.514108739892614\\
0.0808080808080808	-0.462774238781042\\
0.121212121212121	-0.411106480187378\\
0.161616161616162	-0.360142994186573\\
0.202020202020202	-0.309777376484279\\
0.242424242424242	-0.261926891649928\\
0.282828282828283	-0.215545517335396\\
0.323232323232323	-0.171722786226628\\
0.363636363636364	-0.129958633983275\\
0.404040404040404	-0.091770756148711\\
0.444444444444444	-0.0568864370116248\\
0.484848484848485	-0.0254650892622493\\
0.525252525252525	0.00162279602255001\\
0.565656565656566	0.0254226816729527\\
0.606060606060606	0.0446329372227938\\
0.646464646464647	0.0597916316706718\\
0.686868686868687	0.0698841131647441\\
0.727272727272727	0.0754056671001457\\
0.767676767676768	0.0778735579178115\\
0.808080808080808	0.0746918469819204\\
0.848484848484849	0.0669473718129064\\
0.888888888888889	0.055429335004239\\
0.929292929292929	0.0392303872472204\\
0.96969696969697	0.0183827790790897\\
1.01010101010101	-0.00650161266573239\\
1.05050505050505	-0.0351859923025256\\
1.09090909090909	-0.0674221442327219\\
1.13131313131313	-0.103254200247287\\
1.17171717171717	-0.14138847065549\\
1.21212121212121	-0.182530097887147\\
1.25252525252525	-0.225942090394648\\
1.29292929292929	-0.271532683287144\\
1.33333333333333	-0.319282389956706\\
1.37373737373737	-0.367594923145538\\
1.41414141414141	-0.417176856274559\\
1.45454545454545	-0.466821717768883\\
1.4949494949495	-0.517042794124366\\
1.53535353535354	-0.565777420861118\\
1.57575757575758	-0.614127900692444\\
1.61616161616162	-0.661128377619112\\
1.65656565656566	-0.707210911488049\\
1.6969696969697	-0.751012007246147\\
1.73737373737374	-0.792062073894395\\
1.77777777777778	-0.830744960394055\\
1.81818181818182	-0.865510277763062\\
1.85858585858586	-0.89784415458274\\
1.8989898989899	-0.925452553745566\\
1.93939393939394	-0.950172615286139\\
1.97979797979798	-0.971113978079848\\
2.02020202020202	-0.986907060721273\\
2.06060606060606	-0.999270315171331\\
2.1010101010101	-1.00716423462709\\
2.14141414141414	-1.0103392581545\\
2.18181818181818	-1.01007672257376\\
2.22222222222222	-1.00484149186727\\
2.26262626262626	-0.996534897647991\\
2.3030303030303	-0.983696514487684\\
2.34343434343434	-0.967478444352374\\
2.38383838383838	-0.948597225751474\\
2.42424242424242	-0.925466827561425\\
2.46464646464646	-0.900488883575632\\
2.50505050505051	-0.873321925209953\\
2.54545454545455	-0.843768511038713\\
2.58585858585859	-0.81254709031918\\
2.62626262626263	-0.779493652289123\\
2.66666666666667	-0.747450952745425\\
2.70707070707071	-0.713499504911219\\
2.74747474747475	-0.680661682946339\\
2.78787878787879	-0.648509373541471\\
2.82828282828283	-0.616373663223077\\
2.86868686868687	-0.585336614389221\\
2.90909090909091	-0.556775299103559\\
2.94949494949495	-0.529987859349053\\
2.98989898989899	-0.505820873063858\\
3.03030303030303	-0.483434809212831\\
3.07070707070707	-0.465320829091471\\
3.11111111111111	-0.449176248961315\\
3.15151515151515	-0.436467048210297\\
3.19191919191919	-0.426365939816395\\
3.23232323232323	-0.420920342955143\\
3.27272727272727	-0.41833463362748\\
3.31313131313131	-0.419201893268313\\
3.35353535353535	-0.42403357070526\\
3.39393939393939	-0.431835137605278\\
3.43434343434343	-0.443235655585868\\
3.47474747474747	-0.458690381952426\\
3.51515151515152	-0.476266402814265\\
3.55555555555556	-0.497028528384847\\
3.5959595959596	-0.520041894802794\\
3.63636363636364	-0.546193162579345\\
3.67676767676768	-0.574920722196065\\
3.71717171717172	-0.605373217236675\\
3.75757575757576	-0.63768286135558\\
3.7979797979798	-0.671909925388647\\
3.83838383838384	-0.707299181942772\\
3.87878787878788	-0.744392535200946\\
3.91919191919192	-0.780940909695827\\
3.95959595959596	-0.819837646879519\\
4	-0.85904714949231\\
};
\addplot [color=mycolor1, forget plot]
  table[row sep=crcr]{%
0	-0.285867400672155\\
0.0404040404040404	-0.272118989697362\\
0.0808080808080808	-0.258467329095507\\
0.121212121212121	-0.244748527980319\\
0.161616161616162	-0.230966677762593\\
0.202020202020202	-0.219311052599754\\
0.242424242424242	-0.207228857367613\\
0.282828282828283	-0.196217882896366\\
0.323232323232323	-0.18480706296104\\
0.363636363636364	-0.173670616118757\\
0.404040404040404	-0.162897780150402\\
0.444444444444444	-0.152898992940239\\
0.484848484848485	-0.143078707996737\\
0.525252525252525	-0.132327346234817\\
0.565656565656566	-0.122606597243919\\
0.606060606060606	-0.112095364662354\\
0.646464646464647	-0.102149448920994\\
0.686868686868687	-0.0918286512230583\\
0.727272727272727	-0.0817303756986829\\
0.767676767676768	-0.0705071875311501\\
0.808080808080808	-0.0594090437817102\\
0.848484848484849	-0.0479484751870198\\
0.888888888888889	-0.0355252186582522\\
0.929292929292929	-0.0236128977376621\\
0.96969696969697	-0.0101591400054417\\
1.01010101010101	0.00362550583860423\\
1.05050505050505	0.0175225444663276\\
1.09090909090909	0.031693434365185\\
1.13131313131313	0.0464462838919699\\
1.17171717171717	0.0613216371521259\\
1.21212121212121	0.076198415242817\\
1.25252525252525	0.0911709084248568\\
1.29292929292929	0.107001752640858\\
1.33333333333333	0.121744347798983\\
1.37373737373737	0.13604776148177\\
1.41414141414141	0.150468004118921\\
1.45454545454545	0.164561306436534\\
1.4949494949495	0.177175112707655\\
1.53535353535354	0.189603185895973\\
1.57575757575758	0.200642679828219\\
1.61616161616162	0.209633963662385\\
1.65656565656566	0.218508265765971\\
1.6969696969697	0.225257717019086\\
1.73737373737374	0.230015767725284\\
1.77777777777778	0.233636080565998\\
1.81818181818182	0.234538048784154\\
1.85858585858586	0.23440486278647\\
1.8989898989899	0.232219158245329\\
1.93939393939394	0.227219106501439\\
1.97979797979798	0.220679812454071\\
2.02020202020202	0.211977206298882\\
2.06060606060606	0.200618309688998\\
2.1010101010101	0.187778832179927\\
2.14141414141414	0.172472240541488\\
2.18181818181818	0.154750290956734\\
2.22222222222222	0.135587941461665\\
2.26262626262626	0.114078984879201\\
2.3030303030303	0.0909677783384167\\
2.34343434343434	0.0657070413405113\\
2.38383838383838	0.0395039054796812\\
2.42424242424242	0.0104204647054385\\
2.46464646464646	-0.0189433987285856\\
2.50505050505051	-0.05035034464147\\
2.54545454545455	-0.0828261697716573\\
2.58585858585859	-0.116157192980805\\
2.62626262626263	-0.150574928077686\\
2.66666666666667	-0.186670175659285\\
2.70707070707071	-0.222742732272632\\
2.74747474747475	-0.25929439250598\\
2.78787878787879	-0.297809294906584\\
2.82828282828283	-0.334900476366661\\
2.86868686868687	-0.373438306787913\\
2.90909090909091	-0.41271381033776\\
2.94949494949495	-0.451513163129597\\
2.98989898989899	-0.489899888606573\\
3.03030303030303	-0.528946676479573\\
3.07070707070707	-0.568204029006601\\
3.11111111111111	-0.606265696752447\\
3.15151515151515	-0.644300537433715\\
3.19191919191919	-0.683023241934236\\
3.23232323232323	-0.719778260821244\\
3.27272727272727	-0.757242077055686\\
3.31313131313131	-0.79293448915517\\
3.35353535353535	-0.828798182696107\\
3.39393939393939	-0.863201610263323\\
3.43434343434343	-0.896613314535078\\
3.47474747474747	-0.929122329122102\\
3.51515151515152	-0.960775163421511\\
3.55555555555556	-0.98980237164395\\
3.5959595959596	-1.01814261414307\\
3.63636363636364	-1.04555543174607\\
3.67676767676768	-1.07106780359462\\
3.71717171717172	-1.09443911231687\\
3.75757575757576	-1.1156839944359\\
3.7979797979798	-1.13477902528097\\
3.83838383838384	-1.15178909153138\\
3.87878787878788	-1.16619562284521\\
3.91919191919192	-1.17810247208295\\
3.95959595959596	-1.18706582150466\\
4	-1.19372825641123\\
};
\addplot [color=mycolor1, forget plot]
  table[row sep=crcr]{%
0	-0.0190978623415843\\
0.0404040404040404	-0.0385192441015073\\
0.0808080808080808	-0.0575046829794307\\
0.121212121212121	-0.0745830061139251\\
0.161616161616162	-0.0915618484710749\\
0.202020202020202	-0.106408431332961\\
0.242424242424242	-0.118835773448257\\
0.282828282828283	-0.130032292701679\\
0.323232323232323	-0.139247740575274\\
0.363636363636364	-0.146750104770694\\
0.404040404040404	-0.151779714460658\\
0.444444444444444	-0.153642345340461\\
0.484848484848485	-0.15401264803292\\
0.525252525252525	-0.152167077789617\\
0.565656565656566	-0.147703915899505\\
0.606060606060606	-0.140413446759926\\
0.646464646464647	-0.131733850042435\\
0.686868686868687	-0.121888182257699\\
0.727272727272727	-0.109753631688986\\
0.767676767676768	-0.0961081724376102\\
0.808080808080808	-0.0809193467645191\\
0.848484848484849	-0.0651423453768554\\
0.888888888888889	-0.0480956644906804\\
0.929292929292929	-0.0304294361008793\\
0.96969696969697	-0.0135161759350173\\
1.01010101010101	0.00469827926537546\\
1.05050505050505	0.0218065650480689\\
1.09090909090909	0.0379497044117143\\
1.13131313131313	0.0541965225015853\\
1.17171717171717	0.0688301566382388\\
1.21212121212121	0.0832130898212723\\
1.25252525252525	0.0942759888981365\\
1.29292929292929	0.105211345959732\\
1.33333333333333	0.114068132907751\\
1.37373737373737	0.121076769253077\\
1.41414141414141	0.125476940411214\\
1.45454545454545	0.128920973386415\\
1.4949494949495	0.129192166191149\\
1.53535353535354	0.12755795883832\\
1.57575757575758	0.123512807104148\\
1.61616161616162	0.117999815230665\\
1.65656565656566	0.109840720398278\\
1.6969696969697	0.100719524892051\\
1.73737373737374	0.0897513211422939\\
1.77777777777778	0.076045988968273\\
1.81818181818182	0.0609521367800316\\
1.85858585858586	0.045256531273038\\
1.8989898989899	0.0275694707820839\\
1.93939393939394	0.00814911752379183\\
1.97979797979798	-0.0104600899302002\\
2.02020202020202	-0.0314444930295096\\
2.06060606060606	-0.052586754626698\\
2.1010101010101	-0.0746557380569077\\
2.14141414141414	-0.0969138641057968\\
2.18181818181818	-0.119242692443032\\
2.22222222222222	-0.141821398382267\\
2.26262626262626	-0.165035021023734\\
2.3030303030303	-0.187292728229217\\
2.34343434343434	-0.210897575935451\\
2.38383838383838	-0.233466518437631\\
2.42424242424242	-0.255715750903378\\
2.46464646464646	-0.277483453636579\\
2.50505050505051	-0.29880352502954\\
2.54545454545455	-0.320117651473251\\
2.58585858585859	-0.340584216946864\\
2.62626262626263	-0.360498968202725\\
2.66666666666667	-0.379908343562546\\
2.70707070707071	-0.397886582025535\\
2.74747474747475	-0.415562370646304\\
2.78787878787879	-0.431906132505109\\
2.82828282828283	-0.447888104868561\\
2.86868686868687	-0.462246522229637\\
2.90909090909091	-0.475231399268319\\
2.94949494949495	-0.486904008731832\\
2.98989898989899	-0.497634589441188\\
3.03030303030303	-0.506979814424723\\
3.07070707070707	-0.513750513355012\\
3.11111111111111	-0.519414625473202\\
3.15151515151515	-0.523307892736304\\
3.19191919191919	-0.526097463964605\\
3.23232323232323	-0.525892444448228\\
3.27272727272727	-0.523804910715544\\
3.31313131313131	-0.519714572315827\\
3.35353535353535	-0.513568512443777\\
3.39393939393939	-0.505370352405244\\
3.43434343434343	-0.4953572895373\\
3.47474747474747	-0.482055757842864\\
3.51515151515152	-0.467993682390588\\
3.55555555555556	-0.451526579470989\\
3.5959595959596	-0.43215372262225\\
3.63636363636364	-0.411641517101743\\
3.67676767676768	-0.389233972304336\\
3.71717171717172	-0.365122063518602\\
3.75757575757576	-0.339512319448925\\
3.7979797979798	-0.312628850282727\\
3.83838383838384	-0.284071550560596\\
3.87878787878788	-0.255326127118656\\
3.91919191919192	-0.22480930837117\\
3.95959595959596	-0.194831771989978\\
4	-0.163577687921916\\
};
\addplot [color=mycolor1, forget plot]
  table[row sep=crcr]{%
0	0.80311677122615\\
0.0404040404040404	0.782402025514348\\
0.0808080808080808	0.760508164491677\\
0.121212121212121	0.735814858477193\\
0.161616161616162	0.709131910790521\\
0.202020202020202	0.681026926654759\\
0.242424242424242	0.651099131396333\\
0.282828282828283	0.619909858597122\\
0.323232323232323	0.586830298926441\\
0.363636363636364	0.551803295582057\\
0.404040404040404	0.516614464489122\\
0.444444444444444	0.47935447396021\\
0.484848484848485	0.442183105532866\\
0.525252525252525	0.403552631787562\\
0.565656565656566	0.365601831595533\\
0.606060606060606	0.326862072550133\\
0.646464646464647	0.288074275094805\\
0.686868686868687	0.249965909140111\\
0.727272727272727	0.213815288339591\\
0.767676767676768	0.17733457057811\\
0.808080808080808	0.142309962268995\\
0.848484848484849	0.108502467630322\\
0.888888888888889	0.0775461708076532\\
0.929292929292929	0.0473535831907002\\
0.96969696969697	0.0196900443882597\\
1.01010101010101	-0.00651025381799452\\
1.05050505050505	-0.0299421379157378\\
1.09090909090909	-0.0509516143545728\\
1.13131313131313	-0.068994366716574\\
1.17171717171717	-0.0854131803124073\\
1.21212121212121	-0.0989112520805658\\
1.25252525252525	-0.111094903362152\\
1.29292929292929	-0.120362683376802\\
1.33333333333333	-0.126646910087504\\
1.37373737373737	-0.131625417764816\\
1.41414141414141	-0.134533966831589\\
1.45454545454545	-0.135694518286911\\
1.4949494949495	-0.135541305261122\\
1.53535353535354	-0.132877673550394\\
1.57575757575758	-0.130184599338026\\
1.61616161616162	-0.126755558864427\\
1.65656565656566	-0.121946610444639\\
1.6969696969697	-0.116578616087576\\
1.73737373737374	-0.111873341000119\\
1.77777777777778	-0.105735251086582\\
1.81818181818182	-0.101261458142853\\
1.85858585858586	-0.0965641643777009\\
1.8989898989899	-0.0932665062090695\\
1.93939393939394	-0.0896421698077067\\
1.97979797979798	-0.0878297315800833\\
2.02020202020202	-0.0872180201834792\\
2.06060606060606	-0.0880632104654035\\
2.1010101010101	-0.0901903960503215\\
2.14141414141414	-0.0940422274537003\\
2.18181818181818	-0.0993038700412503\\
2.22222222222222	-0.106920027596804\\
2.26262626262626	-0.115224266367766\\
2.3030303030303	-0.126125368898368\\
2.34343434343434	-0.138592838530276\\
2.38383838383838	-0.152033142794576\\
2.42424242424242	-0.16834335289465\\
2.46464646464646	-0.185007512614481\\
2.50505050505051	-0.203742729969865\\
2.54545454545455	-0.224173594676531\\
2.58585858585859	-0.24507095426524\\
2.62626262626263	-0.266682479130389\\
2.66666666666667	-0.290218076831771\\
2.70707070707071	-0.314204160995767\\
2.74747474747475	-0.339489467227251\\
2.78787878787879	-0.363697944857794\\
2.82828282828283	-0.389591409531092\\
2.86868686868687	-0.415772612852926\\
2.90909090909091	-0.4420549663007\\
2.94949494949495	-0.467619162989391\\
2.98989898989899	-0.493196852913796\\
3.03030303030303	-0.519415826888043\\
3.07070707070707	-0.545235736131443\\
3.11111111111111	-0.570612298624968\\
3.15151515151515	-0.594754289630177\\
3.19191919191919	-0.620608054420699\\
3.23232323232323	-0.644410475029619\\
3.27272727272727	-0.668785015981296\\
3.31313131313131	-0.692660826852635\\
3.35353535353535	-0.71572609143664\\
3.39393939393939	-0.739186051866991\\
3.43434343434343	-0.762340421383757\\
3.47474747474747	-0.785046818230767\\
3.51515151515152	-0.807190101674802\\
3.55555555555556	-0.830226502921318\\
3.5959595959596	-0.852814572376193\\
3.63636363636364	-0.875119864673511\\
3.67676767676768	-0.898856114571782\\
3.71717171717172	-0.920857902068385\\
3.75757575757576	-0.944338321806139\\
3.7979797979798	-0.966626564450981\\
3.83838383838384	-0.989867776445214\\
3.87878787878788	-1.01301196975806\\
3.91919191919192	-1.03593696221887\\
3.95959595959596	-1.05942531620162\\
4	-1.08325773783879\\
};
\addplot [color=mycolor1, forget plot]
  table[row sep=crcr]{%
0	-0.341819333324438\\
0.0404040404040404	-0.324044489387299\\
0.0808080808080808	-0.305896348666942\\
0.121212121212121	-0.288013376622918\\
0.161616161616162	-0.270718469101205\\
0.202020202020202	-0.254578517460856\\
0.242424242424242	-0.238544696911389\\
0.282828282828283	-0.223310770093592\\
0.323232323232323	-0.208628899468805\\
0.363636363636364	-0.193843735740763\\
0.404040404040404	-0.180397618483915\\
0.444444444444444	-0.167611701001683\\
0.484848484848485	-0.1545282663757\\
0.525252525252525	-0.141894220134103\\
0.565656565656566	-0.129907241893996\\
0.606060606060606	-0.118490862806842\\
0.646464646464647	-0.106090755936158\\
0.686868686868687	-0.0947141352155342\\
0.727272727272727	-0.0832079101179115\\
0.767676767676768	-0.071823803614474\\
0.808080808080808	-0.0603823284389597\\
0.848484848484849	-0.0481407044684249\\
0.888888888888889	-0.0353529435766418\\
0.929292929292929	-0.0231923696132158\\
0.96969696969697	-0.00897626643828304\\
1.01010101010101	0.00396860118008188\\
1.05050505050505	0.0177314898841251\\
1.09090909090909	0.0317704941912087\\
1.13131313131313	0.0464343752516012\\
1.17171717171717	0.0609145102360715\\
1.21212121212121	0.0772937035075204\\
1.25252525252525	0.0925696431982784\\
1.29292929292929	0.109769302901416\\
1.33333333333333	0.12573388897529\\
1.37373737373737	0.141961336870653\\
1.41414141414141	0.159253434212705\\
1.45454545454545	0.176099008134959\\
1.4949494949495	0.192954635177147\\
1.53535353535354	0.209350993475812\\
1.57575757575758	0.225338129525835\\
1.61616161616162	0.240124061749828\\
1.65656565656566	0.255177050971404\\
1.6969696969697	0.268284726050326\\
1.73737373737374	0.281518745930234\\
1.77777777777778	0.292726680652963\\
1.81818181818182	0.302971539305024\\
1.85858585858586	0.311426493210986\\
1.8989898989899	0.3181177516501\\
1.93939393939394	0.322907758091069\\
1.97979797979798	0.325349340006321\\
2.02020202020202	0.325658690126082\\
2.06060606060606	0.322645094382194\\
2.1010101010101	0.318451205950159\\
2.14141414141414	0.310963811481022\\
2.18181818181818	0.300369863352512\\
2.22222222222222	0.287144121419732\\
2.26262626262626	0.270741147571842\\
2.3030303030303	0.251501005329631\\
2.34343434343434	0.230006253549858\\
2.38383838383838	0.204263032659703\\
2.42424242424242	0.175929486823712\\
2.46464646464646	0.14448029640052\\
2.50505050505051	0.111059050870055\\
2.54545454545455	0.0738245102674004\\
2.58585858585859	0.0342724094455579\\
2.62626262626263	-0.00892334004887413\\
2.66666666666667	-0.053686593010988\\
2.70707070707071	-0.100088254578201\\
2.74747474747475	-0.150345074130726\\
2.78787878787879	-0.202092331071479\\
2.82828282828283	-0.255232554889841\\
2.86868686868687	-0.310265882169227\\
2.90909090909091	-0.367253670835618\\
2.94949494949495	-0.425587579100679\\
2.98989898989899	-0.485188894509226\\
3.03030303030303	-0.544906772677319\\
3.07070707070707	-0.607583471584791\\
3.11111111111111	-0.670105143695873\\
3.15151515151515	-0.732141340407884\\
3.19191919191919	-0.795505451364447\\
3.23232323232323	-0.859659645575136\\
3.27272727272727	-0.923128918897716\\
3.31313131313131	-0.986330610273581\\
3.35353535353535	-1.05059575431863\\
3.39393939393939	-1.1142340743646\\
3.43434343434343	-1.17693145522912\\
3.47474747474747	-1.23884846618317\\
3.51515151515152	-1.30044713165773\\
3.55555555555556	-1.36086246864811\\
3.5959595959596	-1.41978639120865\\
3.63636363636364	-1.47820390781301\\
3.67676767676768	-1.53467323231047\\
3.71717171717172	-1.59017076585496\\
3.75757575757576	-1.6439919922041\\
3.7979797979798	-1.69534910323032\\
3.83838383838384	-1.74542200525356\\
3.87878787878788	-1.79342038816994\\
3.91919191919192	-1.84005089452457\\
3.95959595959596	-1.88411439321411\\
4	-1.9264988815379\\
};
\addplot [color=mycolor1, forget plot]
  table[row sep=crcr]{%
0	0.028134188295961\\
0.0404040404040404	0.0494240533158098\\
0.0808080808080808	0.0687746525539584\\
0.121212121212121	0.0875552470085376\\
0.161616161616162	0.104929127054431\\
0.202020202020202	0.120239208773154\\
0.242424242424242	0.134492320696302\\
0.282828282828283	0.14756759377867\\
0.323232323232323	0.158951276486397\\
0.363636363636364	0.16836806965657\\
0.404040404040404	0.174880939952109\\
0.444444444444444	0.179826953026377\\
0.484848484848485	0.182739057983932\\
0.525252525252525	0.182398934774362\\
0.565656565656566	0.180528028342484\\
0.606060606060606	0.175381418730342\\
0.646464646464647	0.168456488498194\\
0.686868686868687	0.158659750359848\\
0.727272727272727	0.146055123307871\\
0.767676767676768	0.131105509530811\\
0.808080808080808	0.113876647493278\\
0.848484848484849	0.0936331092969582\\
0.888888888888889	0.0718484034843462\\
0.929292929292929	0.0470518746417278\\
0.96969696969697	0.0207995516411832\\
1.01010101010101	-0.0072777596324819\\
1.05050505050505	-0.0370603935446412\\
1.09090909090909	-0.0689952402972381\\
1.13131313131313	-0.10271501204863\\
1.17171717171717	-0.137020523123303\\
1.21212121212121	-0.17157733405021\\
1.25252525252525	-0.207549615051973\\
1.29292929292929	-0.244215370905884\\
1.33333333333333	-0.280800549003491\\
1.37373737373737	-0.318145321423975\\
1.41414141414141	-0.355154337892914\\
1.45454545454545	-0.391756438455836\\
1.4949494949495	-0.427410670659241\\
1.53535353535354	-0.462695897913777\\
1.57575757575758	-0.495565272990412\\
1.61616161616162	-0.528103961510604\\
1.65656565656566	-0.559165688097782\\
1.6969696969697	-0.588411236847104\\
1.73737373737374	-0.616425397868193\\
1.77777777777778	-0.642978119046304\\
1.81818181818182	-0.667508051378994\\
1.85858585858586	-0.68865593332523\\
1.8989898989899	-0.709082004580737\\
1.93939393939394	-0.727619126706512\\
1.97979797979798	-0.743653325792565\\
2.02020202020202	-0.757473895684466\\
2.06060606060606	-0.768818582861251\\
2.1010101010101	-0.777884953837624\\
2.14141414141414	-0.785165698235514\\
2.18181818181818	-0.790066252133725\\
2.22222222222222	-0.793339271537878\\
2.26262626262626	-0.794078530484371\\
2.3030303030303	-0.792493269123125\\
2.34343434343434	-0.789047445810304\\
2.38383838383838	-0.783863603385444\\
2.42424242424242	-0.776751943345574\\
2.46464646464646	-0.768130759599567\\
2.50505050505051	-0.756408346043308\\
2.54545454545455	-0.744242870269088\\
2.58585858585859	-0.730059208068823\\
2.62626262626263	-0.713685843317302\\
2.66666666666667	-0.696600913319617\\
2.70707070707071	-0.677475329074426\\
2.74747474747475	-0.657115460622597\\
2.78787878787879	-0.63523607013205\\
2.82828282828283	-0.611935386344543\\
2.86868686868687	-0.587064884307121\\
2.90909090909091	-0.5613723685412\\
2.94949494949495	-0.534838512395706\\
2.98989898989899	-0.506617154420695\\
3.03030303030303	-0.478217283902595\\
3.07070707070707	-0.448912276814505\\
3.11111111111111	-0.418731100345778\\
3.15151515151515	-0.387745959207492\\
3.19191919191919	-0.356287186088397\\
3.23232323232323	-0.323968681126783\\
3.27272727272727	-0.291266040039517\\
3.31313131313131	-0.259027251238641\\
3.35353535353535	-0.225811489339694\\
3.39393939393939	-0.192781745711457\\
3.43434343434343	-0.159042578713003\\
3.47474747474747	-0.126045972483667\\
3.51515151515152	-0.0916623095112543\\
3.55555555555556	-0.0582571238516614\\
3.5959595959596	-0.0246526005469165\\
3.63636363636364	0.00946926643805512\\
3.67676767676768	0.0428689915204217\\
3.71717171717172	0.0758813650552072\\
3.75757575757576	0.109799536764039\\
3.7979797979798	0.14362766707186\\
3.83838383838384	0.177422480254904\\
3.87878787878788	0.210870275616452\\
3.91919191919192	0.244687146287007\\
3.95959595959596	0.277994923686416\\
4	0.312244853707433\\
};
\end{axis}
\end{tikzpicture}%
			\caption{Gaussian process}
			\label{fig:gaussian:proc}
		\end{subfigure}%
		\caption{Probabilistic applications of Gaussians}
		\medskip
		\small
		The grey shaded region shows a \SI{90}{\percent} confidence interval, and red markers show samples taken from the distribution. Above 1D, plotting the probability density becomes impractical.
	\end{figure}

	A Gaussian (or \enquote{normal}) distribution (\cref{fig:gaussian:1d})is described by a mean $\mu$ and a variance $\sigma^2$, and satisfies
	\begin{align}
		x &\sim \mathcal{N}(\mu, \sigma) &\implies
		p(x; \mu, \sigma^2) &= \frac{1}{\sqrt{2\pi\sigma^2}} \exp{\left[
			-\frac{1}{2} \frac{(x - \mu)^2}{\sigma^2}
		\right]}\,.
	\intertext{
	This describes only a single variable.
	It can be extended to the multivariate Gaussian distribution over $\bm{x} \in \mathbb{R}^d$ (\cref{fig:gaussian:2d}), which is described by a mean \emph{vector} $\bm{\mu}$ and a \emph{co}variance \emph{matrix} $\Sigma$, satisfying
	}
		\bm{x} &\sim \mathcal{N}(\bm{\mu}, \Sigma) &\implies
		p(\bm{x}; \bm{\mu}, \Sigma)
			&= \frac{1}{\sqrt{(2\pi)^d|\Sigma|}} \exp{\left[
				-\frac{1}{2} (\bm{x} - \bm{\mu})^T \Sigma^{-1} (\bm{x} - \bm{\mu})
			\right]}\,.
	\end{align}
	This is able to describe correlations between the finite set of variables $x_i, i\in [1,d]$.

	Gaussian processes are a generalization of multivariate Gaussians to infinitely many variables.
	This is done by noting that the vector $\mathbf{x} \in \mathbb{R}^k$ can be re-expressed as a discrete function of the index $x\colon [1,d] \to \mathbb{R}$, where $x_i = x(i)$.
	From there, the domain can be expanded to the reals to produce a continuous scalar function, such that $x\colon \mathbb{R} \to \mathbb{R}$.
	Gaussian processes (\cref{fig:gaussian:proc}) are described by a mean \emph{function} $m(i)$ and a covariance \emph{function} $K(i, j)$, satisfying
	\begin{align}
		x(\cdot) &\sim \mathrm{GP}(m(\cdot), K(\cdot,\cdot)) \\\implies
		p(x(\cdot); m(\cdot), K(\cdot,\cdot))
			&\propto \exp{\left[
				-\frac{1}{2}
				\iint
					\left(x(i) - m(i)\right)
					{K(i,j)}^{-1}
					\left(x(j) - m(j)\right)
				\diff i \diff j
			\right]}\,.
	\end{align}

	We can take our generalization one further still, by replacing the scalar function $x\colon \mathbb{R} \to \mathbb{R}$ with a scalar field $x\colon \mathbb{R}^m \to \mathbb{R}$.
	With this modification, the only change to the above equation is that the scalars $i,j$ become the vectors $\bm{i}$, $\bm{j}$.
	By defining
	\begin{align}
		\bm{z}\tind{i} &= \begin{bmatrix}\bm{x}\tind{i} \\ \bm{u}\tind{i}\end{bmatrix} &
		\Delta\bm{x}\tind{i} &= \bm{x}_{i + 1} - \bm{x}\tind{i}\,,
	\end{align}
	we can express \cref{eq:transition} in terms of a series of convenient $\mathbb{R}^m \to \mathbb{R}$ functions $f_j$ that can be modelled by such a distribution
	\begin{align}
		\Delta\bm{x}\tind{i}_j &= f_j(\bm{z}\tind{i}) & f_j &\sim \mathrm{GP}(m_j(\bm{z}), K_j(\bm{z}_1, \bm{z}_2))\,, \label{eq:transition-gp}
	\end{align}
	giving a probabilistic representation of discrete system dynamics.
	Such a representation is able to capture model uncertainty, process noise, and observation noise.

\section{The \textsc{Pilco} approach}

	Armed with a probabilistic representation of dynamics, \citeauthor{pilco}'s \textsc{Pilco} \cite{pilco} (Probabilistic Inference for Learning Control) provides a technique to learn the parameters of this representation, and to select a probabilistically optimal controller.

	As such, the optimization in \cref{eq:optimal} also becomes probabilistic\footnotemark, with $\bm{x}\tind{i}$ and $\bm{u}\tind{i}$ becoming distributions over, not simply values of, states and outputs. These distributions can be found by sequentially applying the probabilistic one-step transition equation described in \cref{eq:transition-gp}.

	\footnotetext{
		To make the optimization well-defined, $J$ needs to be chosen to collapse the distribution back down into a scalar. A simple choice is to define $J'(...) = \mathbb{E}_{...}[J(...)]$, but more powerful options exist, such as penalizing uncertainty in the cost.
	}

	The learning process consists of the following steps, repeated cyclically until a sufficient controller is obtained:
	\begin{enumerate}[nosep]
		\item Choose an initial control policy
		\item Apply the latest policy to the robot (perform a \enquote{rollout}), recording state and action trajectories \label{list:pilco:rollout}
		\item Train the probabilistic dynamics model from all past data
		\item Choose the policy that minimizes the cost $J$ when applied to a system with the newly-learnt dynamics, and return to step \ref{list:pilco:rollout}
	\end{enumerate}
	In practice, the first time that step \ref{list:pilco:rollout} is executed, multiple rollouts will be done, in order to provide enough data to sensibly train the dynamics model.

	The key advantage of this method is that it requires orders of magnitude less interaction time \cite{pilco} than competing methods -– important for systems like a unicycle, where each failed interaction (falling over) can cause damage.

\section{Unicycle robots}

	Unicycles provide a challenging control problem to human riders, so intuitively provide a good testing ground for new control techniques.
	A reasonable approximation for how a human rider operates a unicycle is that they have two degrees of freedom -- one through the pedals, and the other by adjusting their angular rotation through the vertical axis.
	This is mirrored in the design of the robotic unicycle, which has one motor attached to the drive wheel, and the other attached to a vertical flywheel\footnotemark (henceforth referred to as the turntable).
	The state space of such a unicycle is $\mathbb{R}^{12}$ \cite{forster}, so this system is heavily underactuated, explaining the difficulty in controlling it.

	\footnotetext{While it may seem unfair that flywheel can rotate freely, while a human is limited to about \SI{180}{\degree} of motion, this intuition doesn't tell the whole story -- human riders are able to change their moment of inertia by extending and contracting their arms.}


	There is a history of unicycle robots in the Engineering Department dating back to 2005, which is described in more detail by \citeauthor{queiro} \cite{queiro}.
	Alongside work on these, the \textsc{Pilco} method has been successfully applied to computer models of these unicycles  \cite[section~3.3]{pilco}.
	Based on the work in 2011, the decision was made to move from the large and dangerous platform described there to a much smaller and safer model.
	Since then, work by \citeauthor{aleksi} \cite{aleksi} went on to write the embedded software for the system, and performed experiments to try and reproduce the results of \textsc{Pilco} in simulation in hardware.

	This work met mixed success -- many problems in the hardware were identified and fixed, and there was some evidence of learning, with the controller marginally improving over time.
	However, various concerns were raised with the testing procedure, and the computer model of the unicycle was never updated to match the hardware -- making it difficult to judge whether problems lay in the hardware or in simply a more difficult control task than on the larger unicycle, and raising concerns about whether other parts of the software stack were still configured for the large robot.

	\subsection{Application of \textsc{Pilco} to the unicycle}
		For the purposes of \textsc{Pilco}, our action and reduced (ie., the components used for learning) state vectors are
		\begin{align}
			\bm{x} &= \begin{bmatrix}
				\dot\theta & \dot\phi &\dot\psi_w & \dot\psi & \dot\psi_t &
				x_c & y_c &
				\theta &
				\phi & \psi
			\end{bmatrix}^T &
			\bm{u} &= \begin{bmatrix}
				\tau_t & \tau_w
			\end{bmatrix}^T\,, \label{eq:state-vars}
		\end{align}
		where $\theta$ is the roll angle, $\phi$ is yaw, $\psi_w$ wheel angle,
		$\psi$ pitch angle, $\psi_t$ the turntable angle, and $x_c, y_c$ are the position of the world-origin in the coordinate space of the robot.
		$\tau_t$ and $\tau_w$ are the control torques on the turntable and wheel, respectively.
		In simulation, some extra states ($\dot{x}_c, \dot{y}_c, \psi_w, \psi_t$) are needed in order to implement the dynamics derived by \citeauthor{forster} \cite{forster}.

		In this report, we restrict our search for optimal controllers to affine controllers of the form
		$\bm{u} = \pi(\bm{x}) = W\bm{x} + \bm{b}$, and choose cost functions of the form
		$c(\bm{x}, \bm{u}) = 1 - \mathbb{E}_{\bm{x}, \bm{u}} \exp\left[-\frac{1}{2} f(\bm{x})^T Q f(\bm{x})\right]$, where $f$ is a function that appends trigonometric functions of $\phi, \theta, \psi$ to the end of the state vector, aiding in penalization of geometric properties.

		Our GP dynamics model is parametrized by $m$, $K$ of the form
		\begin{align}
			m(\bm{z}) &= \bm{z} \cdot \bm{w} + b \\
			K(\bm{z}_1, \bm{z}_2) &= \sigma_s^2 \exp \left(
				-\frac{1}{2}
				\bm{z}_1^T
				\begin{bmatrix}
					\bm{l}_1^2 && \\
					& \ddots & \\
					&& \bm{l}_n^2
				\end{bmatrix}^{-1}
				\bm{z}_2
			\right) + \sigma_n^2 I \,,
		\end{align}
		where $\sigma_s^2$ and $\sigma_n^2$ estimate the signal and noise variances, $w$ (weight) and $b$ (bias) define a simple affine mean function, and $\bm{l}$ (length) defines a length-scale for the problem for each of the state variables.

\bib

\end{document}