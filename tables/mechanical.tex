{
\small
\renewcommand{\arraystretch}{1.5}
\begin{tabularx}{\linewidth}{
	@{}
	r
	>{\raggedright}p{0.25\linewidth}
	S[
		scientific-notation = engineering,
		table-format=-3e-1,
		table-auto-round
	]
	s
	X
	@{}
}
\toprule
 & Description & \multicolumn{2}{c}{Value} & Determined by \\
\midrule
	\texttt{mt}
		& Mass of turntable (tt.)
		& 0.2110 & \kilogram
		& estimation using $\rho_\mathrm{steel} = 8000\si{\kilogram\per\cubic\meter}$.
	\\
	\texttt{mw}
		& Mass of wheel and axle
		& 0.1090 & \kilogram
		& direct measurement, by detaching the axle bearings, and suspending the robot frame such that it does not exert any reaction force on the axle.
	\\
	\texttt{mf}
		& Mass of frame
		& 0.6600 & \kilogram
		& subtracting from the total mass of \SI{0.98}{\kilogram}.
	\\
\midrule
	\texttt{rw}
		& Radius of wheel
		& 0.0353 & \meter
		& measuring the wheel circumference as \SI{22.2}{\centi\meter}.
	\\
	\texttt{rf}
		& Distance from COM of frame to wheel
		& 0.0800 & \meter
		&\tikzmark{bra1}\multirow{3}{\hsize}{%
			finding the balancing point of the robot, by suspending from a string.
		}
	\\
	\texttt{rt}
		& Distance from COM of frame to turntable
		& 0.0070 & \meter
		& \tikzmark{brb1}
	\\
\midrule
	\texttt{Cw}
		& MOI of wheel $\parallel$ to axle
		& 3.0536e-07 & \kilogram \square\meter
		&\tikzmark{bra2}\multirow{2}{\hsize}{%
			computing $m_\mathrm{wheel} = \mathtt{mw} - \rho_\mathrm{steel} \pi \allowbreak(lr^2)_\textrm{axle}$, and then assuming a uniform laminar disc.
		}
	\\
	\texttt{Aw}
		& MOI of wheel $\bot$ to axle
		& 1.2739e-04 & \kilogram \square\meter
		& \tikzmark{brb2}
	\\
	\texttt{Cf}
		& MOI of frame
		& 5.1116e-04 & \kilogram \square\meter
		&\tikzmark{bra3}\multirow{3}{\hsize}{%
			estimation by dimensional analysis, assuming that $I \propto \mathtt{mf}\,\mathtt{rf}^2$, and using known values for the large unicycle.
		}
	\\
	\texttt{Bf}
		& MOI of frame
		& 2.8406e-04 & \kilogram \square\meter
		&
	\\
	\texttt{Af}
		& MOI of frame
		& 2.618e-04 & \kilogram \square\meter
		& \tikzmark{brb3}
	\\
	\texttt{Ct}
		& MOI of tt. $\parallel$ to axle
		& 1.611e-04 & \kilogram \square\meter
		&\tikzmark{bra4}\multirow{2}{\hsize}{%
			decomposition into cylinders, and using $\rho_\mathrm{steel}$.
		}
	\\
	\texttt{At}
		& MOI of tt. $\bot$ to axle
		& 8.102e-05 & \kilogram \square\meter
		& \tikzmark{brb4}
	\\
\midrule
	\texttt{maxU}
		& Maximum input torques
		& {
			\renewcommand{\arraystretch}{1}
			$\begin{bmatrix}0.205 \\ 0.513\end{bmatrix}$
		} & \newton \meter
		& applying the gearing ratios\cite{gearbox} to the torque limits quoted on the datasheet\cite{motor}.
	\\
\bottomrule
\end{tabularx}
\begin{tikzpicture}[overlay, remember picture]
	\draw [decoration={brace,amplitude=0.5em},decorate,thick,black]
		let \p1=(bra1.mid), \p2=(brb1.mid) in
			({\x1 - 0.8em}, {\y1 +0.5em}) -- node[right=0.6em] {} ({\x1 - 0.8em}, {\y2 - 0.5em});
	\draw [decoration={brace,amplitude=0.5em},decorate,thick,black]
		let \p1=(bra2.mid), \p2=(brb2.mid) in
			({\x1 - 0.8em}, {\y1 +0.5em}) -- node[right=0.6em] {} ({\x1 - 0.8em}, {\y2 - 0.5em});
	\draw [decoration={brace,amplitude=0.5em},decorate,thick,black]
		let \p1=(bra3.mid), \p2=(brb3.mid) in
			({\x1 - 0.8em}, {\y1 +0.5em}) -- node[right=0.6em] {} ({\x1 - 0.8em}, {\y2 - 0.5em});
	\draw [decoration={brace,amplitude=0.5em},decorate,thick,black]
		let \p1=(bra4.mid), \p2=(brb4.mid) in
			({\x1 - 0.8em}, {\y1 +0.5em}) -- node[right=0.6em] {} ({\x1 - 0.8em}, {\y2 - 0.5em});
\end{tikzpicture}
}

\centering
\caption{Mechanical properties of the small unicycle}
\medskip
\small
MOI and COM abbreviate Moment of Inertia and Center of Mass, respectively.
The first column indicates the property name of the \texttt{UnicyclePlant} class in the Matlab source code.
Many properties proved impossible to measure due to the inability to disassemble the robot, such as \texttt{mt}, requiring estimation techniques instead.
The working for all of these techniques is implemented in the Matlab \matlab{UnicyclePlant} class.

\medskip
Of these properties, only \texttt{rw} is needed by the embedded software.
To learn a controller on the real robot, only \texttt{rw}, \texttt{rf}, and \texttt{rt} are needed, as these factor into the cost function (\cref{sec:cost-function}).
The full set of properties is only need for complete robot simulation.

