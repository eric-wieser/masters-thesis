% Use documentclass[wide]{IIBproject} to have narrower margins
\documentclass[12pt,a4paper]{IIBproject}

\usepackage[textsize=tiny]{todonotes}
\setlength{\marginparwidth}{2cm}

% vital stuff
\usepackage[utf8]{inputenc}
\usepackage[T1]{fontenc}
\usepackage{subfiles}
\usepackage{ifthen}

% tools used
\usepackage{enumitem}
\usepackage{multicol}
\usepackage{listings}
\usepackage{amsmath}
\usepackage{amsfonts}
\usepackage{bm}
\usepackage{booktabs}
\usepackage{tabularx}
\usepackage{siunitx}
\usepackage{multirow,bigdelim}
\usepackage{subcaption}
\usepackage{calc}
\usepackage{caption}
\captionsetup{format=hang}


% custom commands
\DeclareMathOperator*{\argmin}{arg\,min}
\newcommand*\diff{\mathop{}\!\mathrm{d}}
\newcommand{\cpp}{\lstinline[language={[11]C++}]}
\newcommand{\matlab}{\lstinline[language={[11]C++}]}
\newcommand{\tind}[1]{^{(#1)}}
\newcommand{\unitv}[1]{\hat{\bm{e}}_{#1}}

% for the braces
\usepackage{tikz}
\usetikzlibrary{calc}
\usetikzlibrary{decorations.pathreplacing}
\newcommand{\tikzmark}[1]{\tikz[overlay,remember picture] \node (#1) {};}

% for plots
\usepackage{pgfplots}
\pgfplotsset{compat=newest}
%% the following commands are needed for some matlab2tikz features
\usetikzlibrary{plotmarks}
\usetikzlibrary{arrows.meta}
\usetikzlibrary{decorations.markings}
\usepgfplotslibrary{patchplots}
\usepackage{grffile}
\pgfplotsset{%
	every axis/.append style={
		label style={font=\scriptsize},
		tick label style={font=\scriptsize}
	}
}
\tikzset{
  set arrow inside/.code={\pgfqkeys{/tikz/arrow inside}{#1}},
  set arrow inside={end/.initial=>, opt/.initial=},
  /pgf/decoration/Mark/.style={
    mark/.expanded=at position #1 with
    {
      \noexpand\arrow[\pgfkeysvalueof{/tikz/arrow inside/opt}]{\pgfkeysvalueof{/tikz/arrow inside/end}}
    }
  },
  arrow inside/.style 2 args={
    set arrow inside={#1},
    postaction={
      decorate,decoration={
        markings,Mark/.list={#2}
      }
    }
  },
}

% style only
\usepackage[scaled=0.85]{beramono}
\usepackage[onehalfspacing]{setspace}
\usepackage{parskip}

\lstset{
	basicstyle=\ttfamily\footnotesize,
    keywordstyle=\color[rgb]{0,0,1},
    commentstyle=\color[rgb]{0.026,0.112,0.095},
    stringstyle=\color[rgb]{0.627,0.126,0.941},
    numberstyle=\color[rgb]{0.205, 0.142, 0.73},
	breaklines=true,
	tabsize=2,
	keepspaces=true,
	columns=flexible
}

% related to links and references
\usepackage{xr-hyper}
\usepackage[bookmarks,bookmarksnumbered,bookmarksopen]{hyperref}
\usepackage{cleveref}
\def\UrlBreaks{\do\/\do-}

% custom listing float type
\usepackage{newfloat}
\DeclareFloatingEnvironment[
   name=Listing
]{listingfloat}
\crefname{listingfloat}{listing}{listings}
\Crefname{listingfloat}{Listing}{Listings}

% bibliography stuff
\usepackage[sorting=none,backend=biber,bibencoding=utf8]{biblatex}
\providecommand{\thebibpath}{.}
\addbibresource{\thebibpath/main.bib}
\newcommand{\printbibsectioned}{
	\printbibliography[
		notkeyword={hardware}, notkeyword={software},
		title={References},
		heading=bibintoc]
	\printbibliography[
		notkeyword={hardware}, keyword={software},
		title={Software packages},
		heading=subbibintoc]
	\printbibliography[
		keyword={hardware},
		title={Hardware datasheets},
		heading=subbibintoc]
}
\newboolean{printBibInSubfiles}
\setboolean{printBibInSubfiles}{true}
\def\bib{\ifthenelse{\boolean{printBibInSubfiles}}
        {\printbibsectioned}
        {}
    }


\pagestyle{empty}

\begin{document}
	\setboolean{printBibInSubfiles}{false}


	% Title Page
	\author{Eric Wieser}
	\supervisor{Prof. Carl Rasmussen}
	\title{Machine learning for a miniature robotic unicycle}
	\maketitle
	\todo{Title page is broken with the \texttt{report} class}
	% \thispagestyle{empty}



	% Summary
	\begin{abstract}
	\todo{Write this}
	\end{abstract}
	\pagestyle{plain}
	\tableofcontents
	\newpage

	\listoftodos

	\newpage

	\chapter{Introduction}
	\subfile{sections/intro}

	\chapter{Review of the current system design}
	\subfile{sections/review}

	\chapter{Design changes and new features}

		\section{Electrical design}
		\subfile{sections/electrical}

		\section{Embedded software}
		\subfile{sections/embedded}

		\section{Matlab toolbox}
		\subfile{sections/matlab}

	\chapter{System verification}
	\subfile{sections/verification}

	\chapter{Machine learning experiments}

		\section{Simulation}
		\subfile{sections/simulation}


		\section{Reality}
		\subfile{sections/hardware-experiments}

	\chapter{Future Work}
	\todo[inline]{Write this properly}
	\begin{itemize}
		\item Try automatic differentiation
		\item Fully characterize the orientation of the sensor
		\item Try a quadratic controller
		\item Run more rollouts
		\item Try and encapsulate more of the matlab code
		\item $\cdots$
	\end{itemize}

	\chapter{Conclusions}
	\todo[inline]{Write conclusion}

	\printbibsectioned

	\appendix


	\chapter{Source code repositories}
	\subfile{sections/repos}

	\chapter{Risk assessment retrospective}
	\subfile{sections/risk-assessment}

	\chapter{A primer on quaternions}
	\label{app:quat}
	\subfile{sections/quaternions}

\end{document}
